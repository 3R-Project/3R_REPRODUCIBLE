% Options for packages loaded elsewhere
\PassOptionsToPackage{unicode}{hyperref}
\PassOptionsToPackage{hyphens}{url}
%
\documentclass[
]{article}
\usepackage{amsmath,amssymb}
\usepackage{iftex}
\ifPDFTeX
  \usepackage[T1]{fontenc}
  \usepackage[utf8]{inputenc}
  \usepackage{textcomp} % provide euro and other symbols
\else % if luatex or xetex
  \usepackage{unicode-math} % this also loads fontspec
  \defaultfontfeatures{Scale=MatchLowercase}
  \defaultfontfeatures[\rmfamily]{Ligatures=TeX,Scale=1}
\fi
\usepackage{lmodern}
\ifPDFTeX\else
  % xetex/luatex font selection
    \setmainfont[]{Times New Roman}
\fi
% Use upquote if available, for straight quotes in verbatim environments
\IfFileExists{upquote.sty}{\usepackage{upquote}}{}
\IfFileExists{microtype.sty}{% use microtype if available
  \usepackage[]{microtype}
  \UseMicrotypeSet[protrusion]{basicmath} % disable protrusion for tt fonts
}{}
\makeatletter
\@ifundefined{KOMAClassName}{% if non-KOMA class
  \IfFileExists{parskip.sty}{%
    \usepackage{parskip}
  }{% else
    \setlength{\parindent}{0pt}
    \setlength{\parskip}{6pt plus 2pt minus 1pt}}
}{% if KOMA class
  \KOMAoptions{parskip=half}}
\makeatother
\usepackage{xcolor}
\usepackage[margin=1 in]{geometry}
\usepackage{color}
\usepackage{fancyvrb}
\newcommand{\VerbBar}{|}
\newcommand{\VERB}{\Verb[commandchars=\\\{\}]}
\DefineVerbatimEnvironment{Highlighting}{Verbatim}{commandchars=\\\{\}}
% Add ',fontsize=\small' for more characters per line
\usepackage{framed}
\definecolor{shadecolor}{RGB}{248,248,248}
\newenvironment{Shaded}{\begin{snugshade}}{\end{snugshade}}
\newcommand{\AlertTok}[1]{\textcolor[rgb]{0.94,0.16,0.16}{#1}}
\newcommand{\AnnotationTok}[1]{\textcolor[rgb]{0.56,0.35,0.01}{\textbf{\textit{#1}}}}
\newcommand{\AttributeTok}[1]{\textcolor[rgb]{0.13,0.29,0.53}{#1}}
\newcommand{\BaseNTok}[1]{\textcolor[rgb]{0.00,0.00,0.81}{#1}}
\newcommand{\BuiltInTok}[1]{#1}
\newcommand{\CharTok}[1]{\textcolor[rgb]{0.31,0.60,0.02}{#1}}
\newcommand{\CommentTok}[1]{\textcolor[rgb]{0.56,0.35,0.01}{\textit{#1}}}
\newcommand{\CommentVarTok}[1]{\textcolor[rgb]{0.56,0.35,0.01}{\textbf{\textit{#1}}}}
\newcommand{\ConstantTok}[1]{\textcolor[rgb]{0.56,0.35,0.01}{#1}}
\newcommand{\ControlFlowTok}[1]{\textcolor[rgb]{0.13,0.29,0.53}{\textbf{#1}}}
\newcommand{\DataTypeTok}[1]{\textcolor[rgb]{0.13,0.29,0.53}{#1}}
\newcommand{\DecValTok}[1]{\textcolor[rgb]{0.00,0.00,0.81}{#1}}
\newcommand{\DocumentationTok}[1]{\textcolor[rgb]{0.56,0.35,0.01}{\textbf{\textit{#1}}}}
\newcommand{\ErrorTok}[1]{\textcolor[rgb]{0.64,0.00,0.00}{\textbf{#1}}}
\newcommand{\ExtensionTok}[1]{#1}
\newcommand{\FloatTok}[1]{\textcolor[rgb]{0.00,0.00,0.81}{#1}}
\newcommand{\FunctionTok}[1]{\textcolor[rgb]{0.13,0.29,0.53}{\textbf{#1}}}
\newcommand{\ImportTok}[1]{#1}
\newcommand{\InformationTok}[1]{\textcolor[rgb]{0.56,0.35,0.01}{\textbf{\textit{#1}}}}
\newcommand{\KeywordTok}[1]{\textcolor[rgb]{0.13,0.29,0.53}{\textbf{#1}}}
\newcommand{\NormalTok}[1]{#1}
\newcommand{\OperatorTok}[1]{\textcolor[rgb]{0.81,0.36,0.00}{\textbf{#1}}}
\newcommand{\OtherTok}[1]{\textcolor[rgb]{0.56,0.35,0.01}{#1}}
\newcommand{\PreprocessorTok}[1]{\textcolor[rgb]{0.56,0.35,0.01}{\textit{#1}}}
\newcommand{\RegionMarkerTok}[1]{#1}
\newcommand{\SpecialCharTok}[1]{\textcolor[rgb]{0.81,0.36,0.00}{\textbf{#1}}}
\newcommand{\SpecialStringTok}[1]{\textcolor[rgb]{0.31,0.60,0.02}{#1}}
\newcommand{\StringTok}[1]{\textcolor[rgb]{0.31,0.60,0.02}{#1}}
\newcommand{\VariableTok}[1]{\textcolor[rgb]{0.00,0.00,0.00}{#1}}
\newcommand{\VerbatimStringTok}[1]{\textcolor[rgb]{0.31,0.60,0.02}{#1}}
\newcommand{\WarningTok}[1]{\textcolor[rgb]{0.56,0.35,0.01}{\textbf{\textit{#1}}}}
\usepackage{graphicx}
\makeatletter
\def\maxwidth{\ifdim\Gin@nat@width>\linewidth\linewidth\else\Gin@nat@width\fi}
\def\maxheight{\ifdim\Gin@nat@height>\textheight\textheight\else\Gin@nat@height\fi}
\makeatother
% Scale images if necessary, so that they will not overflow the page
% margins by default, and it is still possible to overwrite the defaults
% using explicit options in \includegraphics[width, height, ...]{}
\setkeys{Gin}{width=\maxwidth,height=\maxheight,keepaspectratio}
% Set default figure placement to htbp
\makeatletter
\def\fps@figure{htbp}
\makeatother
\setlength{\emergencystretch}{3em} % prevent overfull lines
\providecommand{\tightlist}{%
  \setlength{\itemsep}{0pt}\setlength{\parskip}{0pt}}
\setcounter{secnumdepth}{5}
\usepackage{booktabs}
\usepackage{caption}
\usepackage{longtable}
\usepackage{colortbl}
\usepackage{array}
\usepackage{anyfontsize}
\usepackage{multirow}
\ifLuaTeX
  \usepackage{selnolig}  % disable illegal ligatures
\fi
\usepackage{bookmark}
\IfFileExists{xurl.sty}{\usepackage{xurl}}{} % add URL line breaks if available
\urlstyle{same}
\hypersetup{
  pdftitle={REPRODUCIBILIDAD BASADO EN R},
  pdfauthor={3R Project},
  hidelinks,
  pdfcreator={LaTeX via pandoc}}

\title{REPRODUCIBILIDAD BASADO EN R}
\usepackage{etoolbox}
\makeatletter
\providecommand{\subtitle}[1]{% add subtitle to \maketitle
  \apptocmd{\@title}{\par {\large #1 \par}}{}{}
}
\makeatother
\subtitle{REPORTE}
\author{3R Project}
\date{}

\begin{document}
\maketitle

\section{Introducción}\label{introducciuxf3n}

El presente informe forma parte del Semillero de Investigación 3R
Project: Reproducibilidad basada en R. Este proyecto tiene como objetivo
principal transformar investigaciones científicas en documentos
reproducibles, promoviendo la transparencia y la verificabilidad en el
ámbito científico.

En este documento se detalla el proceso de exploración y análisis de un
conjunto de datos, utilizando herramientas de R para garantizar la
reproducibilidad de cada etapa

Durante la revisión de la data y su comparación con el informe original,
se identificaron diversas inconsistencias que requieren un análisis
detallado. Una de las primeras observaciones es que las variables en la
base de datos no están en el mismo idioma que las presentadas en el
informe, lo que puede generar dificultades en la interpretación y
procesamiento de la información.

Asimismo, se detectó una discrepancia en el número total de
observaciones. Según la Tabla 2 del informe, el total de registros
reportados es menor en comparación con los datos almacenados en la base
de datos (Excel). En concreto, la diferencia total es de 134
observaciones menos en el informe, con una reducción específica de 74
registros en España y 60 en Italia.

Otro aspecto crítico identificado es la presencia de registros con
valores faltantes o inválidos en la base de datos, tanto en Excel como
en R. En Excel, se presentan errores como N/A para casillas vacías de
tipo numérico, \#NUM! cuando se intenta calcular logaritmos de números
negativos, casillas en blanco para valores de tipo texto y \#¡DIV/0!
cuando se realizan operaciones con celdas vacías. En R, los errores más
comunes incluyen NAN, que indica casillas vacías, y Inf, que aparece
cuando se intenta realizar un cálculo con valores nulos o se produce un
error en la operación.

Estas observaciones iniciales evidencian la necesidad de llevar a cabo
un proceso de limpieza, estandarización y verificación de la data antes
de continuar con el análisis. Garantizar la coherencia y fiabilidad de
los datos es fundamental para obtener resultados precisos y
reproducibles.

\section{Instalación de Paquetes}\label{instalaciuxf3n-de-paquetes}

Estos paquetes se instalaran para llevar a cabo un flujo de trabajo
reproducible en la exploración de datos. tidyverse permite manipular,
limpiar y visualizar los datos; readr facilita su importación eficiente;
y janitor ayuda a organizarlos y limpiarlos. Para documentar y presentar
los resultados, rstatix realiza análisis estadísticos, mientras que gt
genera tablas profesionales y openxlsx exporta los reportes a Excel. En
conjunto, estos paquetes garantizan un análisis reproducible, claro y
bien documentado.

\begin{Shaded}
\begin{Highlighting}[]
\FunctionTok{install.packages}\NormalTok{(}\StringTok{"tidyverse"}\NormalTok{)}
\FunctionTok{install.packages}\NormalTok{(}\StringTok{"openxlsx"}\NormalTok{)}
\FunctionTok{install.packages}\NormalTok{(}\StringTok{"readr"}\NormalTok{)}
\FunctionTok{install.packages}\NormalTok{(}\StringTok{"janitor"}\NormalTok{)}
\FunctionTok{install.packages}\NormalTok{(}\StringTok{"rstatix"}\NormalTok{)}
\FunctionTok{install.packages}\NormalTok{(}\StringTok{"gt"}\NormalTok{)}
\end{Highlighting}
\end{Shaded}

\section{Carga de Librerias}\label{carga-de-librerias}

Estas líneas de código en R sirven para cargar paquetes previamente
instalados para utilizarlos en el análisis. Cada library() activa el
paquete especificado, permitiendo el acceso a sus funciones. Aquí te
explico brevemente qué hace cada uno:

\begin{Shaded}
\begin{Highlighting}[]
\FunctionTok{library}\NormalTok{(tidyverse)}
\FunctionTok{library}\NormalTok{(dplyr)}
\FunctionTok{library}\NormalTok{(openxlsx)}
\FunctionTok{library}\NormalTok{(readr)}
\FunctionTok{library}\NormalTok{(janitor)}
\FunctionTok{library}\NormalTok{(rstatix)}
\FunctionTok{library}\NormalTok{(gt)}
\end{Highlighting}
\end{Shaded}

\section{Funciones Especificas}\label{funciones-especificas}

Estas son funciones específicas del paquete base stats en R, y están
relacionadas con análisis estadísticos. El prefijo stats:: se usa para
asegurarse de que se está llamando directamente a las funciones del
paquete stats, incluso si hay otras funciones con el mismo nombre en
otros paquetes cargados

\begin{Shaded}
\begin{Highlighting}[]
\NormalTok{stats}\SpecialCharTok{::}\NormalTok{chisq.test}
\NormalTok{stats}\SpecialCharTok{::}\NormalTok{fisher.test}
\NormalTok{stats}\SpecialCharTok{::}\NormalTok{filter }
\end{Highlighting}
\end{Shaded}

\section{Importando Datos}\label{importando-datos}

En esta sección, se carga el conjunto de datos en un objeto de R
utilizando la función read.xlsx. Esto permite explorar y manipular los
datos para análisis posteriores. La finalidad es tener acceso al archivo
original y comenzar con la limpieza y transformación de los datos

\begin{Shaded}
\begin{Highlighting}[]
\FunctionTok{library}\NormalTok{(openxlsx)}
\NormalTok{datos}\OtherTok{\textless{}{-}}\FunctionTok{read.xlsx}\NormalTok{(}\StringTok{"C:/Users/Usuario/Desktop/SEMILLERO/DATA\_STATA/DATA/DatosStata.xlsx"}\NormalTok{)}
\end{Highlighting}
\end{Shaded}

\section{Explorando el Objeto de los
Datos}\label{explorando-el-objeto-de-los-datos}

Aquí se realiza una inspección básica del objeto de datos con la función
str para entender su estructura y tipos de variables.

\begin{Shaded}
\begin{Highlighting}[]
\FunctionTok{str}\NormalTok{(datos)}
\end{Highlighting}
\end{Shaded}

\section{Eliminando Duplicados}\label{eliminando-duplicados}

También se seleccionan y eliminan columnas irrelevantes con select del
paquete dplyr, optimizando el dataset para el análisis. Esto asegura que
solo se trabajen datos útiles para los objetivos del proyecto.

\begin{Shaded}
\begin{Highlighting}[]
\NormalTok{datos\_ex }\OtherTok{\textless{}{-}}\NormalTok{datos }\SpecialCharTok{\%\textgreater{}\%} \FunctionTok{select}\NormalTok{ (}\SpecialCharTok{{-}} \FunctionTok{c}\NormalTok{(Número.de.accionistas,DM.Edad,ADV.Edad))}
\end{Highlighting}
\end{Shaded}

\section{Diccionario de Traducción}\label{diccionario-de-traducciuxf3n}

En esta fase estudio variables del conjunto de datos de la Data que se
encontraba en español. Dado que los datos originales estaban en español,
se identificaron las etiquetas y nombres de las variables que
necesitaban ser traducidas para alinearlas con las utilizadas en el
artículo original, que estaba redactado en inglés. Lo que representaba
un desafío para la comprensión de las variables y su correspondencia con
las mencionadas en el documento fuente. Para garantizar una mayor
claridad y alineación conceptual, se procedió a traducir las etiquetas y
nombres de las variables del archivo Excel del español al inglés. Este
paso fue esencial para comprender mejor las características y contextos
de las variables del artículo y su relación con los datos entregados.

La traducción no solo facilitó la interpretación de los datos, sino que
también permitió identificar posibles discrepancias o inconsistencias
entre la documentación original y los datos proporcionados. Esta etapa
inicial fue fundamental para establecer las bases de un análisis robusto
y coherente con los objetivos del artículo académico.''

\begin{Shaded}
\begin{Highlighting}[]
\NormalTok{DATA }\OtherTok{\textless{}{-}}\NormalTok{ datos\_ex }\SpecialCharTok{\%\textgreater{}\%}
  \FunctionTok{rename}\NormalTok{(}
    \StringTok{"Year"} \OtherTok{=} \StringTok{"Año"}\NormalTok{,}
    \StringTok{"ID"} \OtherTok{=} \StringTok{"Id."}\NormalTok{,}
    \StringTok{"Identity"} \OtherTok{=} \StringTok{"Ident"}\NormalTok{,}
    \StringTok{"Gender"} \OtherTok{=} \StringTok{"Género"}\NormalTok{,}
    \StringTok{"Company\_Name"} \OtherTok{=} \StringTok{"Nombre.empresa"}\NormalTok{,}
    \StringTok{"City"} \OtherTok{=} \StringTok{"Ciudad"}\NormalTok{,}
    \StringTok{"Country\_ISO\_Code"} \OtherTok{=} \StringTok{"Código.ISO.del.país"}\NormalTok{,}
    \StringTok{"Total\_Assets"} \OtherTok{=} \StringTok{"Activos.totales"}\NormalTok{,}
    \StringTok{"Growth"} \OtherTok{=} \StringTok{"Grow"}\NormalTok{,}
    \StringTok{"GDP"} \OtherTok{=} \StringTok{"PIB"}\NormalTok{,}
    \StringTok{"GDP\_Var"} \OtherTok{=} \StringTok{"VarPIB"}\NormalTok{,}
    \StringTok{"Inflation"} \OtherTok{=} \StringTok{"Inflacion"}\NormalTok{,}
    \StringTok{"Ln\_GDP"} \OtherTok{=} \StringTok{"LnPIB"}\NormalTok{,}
    \StringTok{"Ln\_Inflation"} \OtherTok{=} \StringTok{"LnInflacion"}\NormalTok{,}
    \StringTok{"Country"} \OtherTok{=} \StringTok{"Pais"}\NormalTok{,}
    \StringTok{"NACE\_Code"} \OtherTok{=} \StringTok{"NACE.code"}\NormalTok{,}
    \StringTok{"Employees\_Last\_Year"} \OtherTok{=} \StringTok{"Número.empleados.Últ..año.disp."}\NormalTok{,}
    \StringTok{"Standard\_Legal\_Form"} \OtherTok{=} \StringTok{"Forma.jurídica.estándar"}\NormalTok{,}
    \StringTok{"Legal\_Form\_Tabul"} \OtherTok{=} \StringTok{"FJurídicaTabul"}\NormalTok{,}
    \StringTok{"Legal\_Form"} \OtherTok{=} \StringTok{"FormaJurídica"}\NormalTok{,}
    \StringTok{"Incorporation\_Date"} \OtherTok{=} \StringTok{"Fecha.de.constitución"}\NormalTok{,}
    \StringTok{"End\_Date"} \OtherTok{=} \StringTok{"Fecha.final"}\NormalTok{,}
    \StringTok{"Seniority"} \OtherTok{=} \StringTok{"Antigüedad"}\NormalTok{,}
    \StringTok{"Ln\_Seniority"} \OtherTok{=} \StringTok{"LnAntigüedad"}\NormalTok{,}
    \StringTok{"Cash\_Flows"} \OtherTok{=} \StringTok{"Flujos.de.Caja"}\NormalTok{,}
    \StringTok{"Fixed\_Assets"} \OtherTok{=} \StringTok{"Activos.Fijos"}\NormalTok{,}
    \StringTok{"Current\_Assets"} \OtherTok{=} \StringTok{"Activos.Corrientes"}\NormalTok{,}
    \StringTok{"Inventory"} \OtherTok{=} \StringTok{"Stock"}\NormalTok{,}
    \StringTok{"Receivables"} \OtherTok{=} \StringTok{"Deudores"}\NormalTok{,}
    \StringTok{"Other\_Current\_Assets"} \OtherTok{=} \StringTok{"Otros.activos.corrientes"}\NormalTok{,}
    \StringTok{"Cash\_and\_Equivalents"} \OtherTok{=} \StringTok{"Efectivo.y.equivalentes"}\NormalTok{,}
    \StringTok{"Ln\_Fixed\_Assets"} \OtherTok{=} \StringTok{"LnActFijo"}\NormalTok{,}
    \StringTok{"Ln\_Current\_Assets"} \OtherTok{=} \StringTok{"LnActCorr"}\NormalTok{,}
    \StringTok{"Ln\_Inventory"} \OtherTok{=} \StringTok{"LnStock"}\NormalTok{,}
    \StringTok{"Ln\_Receivables"} \OtherTok{=} \StringTok{"LnDeudores"}\NormalTok{,}
    \StringTok{"Ln\_Other\_Assets"} \OtherTok{=} \StringTok{"LnOtrosactiv"}\NormalTok{,}
    \StringTok{"Ln\_Cash"} \OtherTok{=} \StringTok{"LnEfectivo"}\NormalTok{,}
    \StringTok{"Ln\_Total\_Assets"} \OtherTok{=} \StringTok{"LnActTotal"}\NormalTok{,}
    \StringTok{"Non\_Current\_Liabilities"} \OtherTok{=} \StringTok{"Pasivos.no.corrientes"}\NormalTok{,}
    \StringTok{"Current\_Liabilities"} \OtherTok{=} \StringTok{"Pasivos.Corrientes"}\NormalTok{,}
    \StringTok{"Liquidity1"} \OtherTok{=} \StringTok{"Liquidez1"}\NormalTok{,}
    \StringTok{"Liquidity1\_Dummy"} \OtherTok{=} \StringTok{"Liquidez1Dummy"}\NormalTok{,}
    \StringTok{"Total\_Liabilities"} \OtherTok{=} \StringTok{"Pasivo.Total"}\NormalTok{,}
    \StringTok{"Equity"} \OtherTok{=} \StringTok{"Fondos.Propios"}\NormalTok{,}
    \StringTok{"Ln\_Non\_Current\_Liabilities"} \OtherTok{=} \StringTok{"LnPasivoNoCorr"}\NormalTok{,}
    \StringTok{"Ln\_Current\_Liabilities"} \OtherTok{=} \StringTok{"LnPasivoCorr"}\NormalTok{,}
    \StringTok{"Ln\_Total\_Liabilities"} \OtherTok{=} \StringTok{"LnPasivoTotal"}\NormalTok{,}
    \StringTok{"Ln\_Equity"} \OtherTok{=} \StringTok{"LnFondosPropios"}\NormalTok{,}
    \StringTok{"Operating\_Revenue"} \OtherTok{=} \StringTok{"Ingresos.Explotación"}\NormalTok{,}
    \StringTok{"Operating\_Profit"} \OtherTok{=} \StringTok{"Resultado.Explotación"}\NormalTok{,}
    \StringTok{"Financial\_Expenses"} \OtherTok{=} \StringTok{"Gastos.Financieros"}\NormalTok{,}
    \StringTok{"Ordinary\_Profit\_Before\_Tax"} \OtherTok{=} \StringTok{"Rdo..Ordinario.antes.Impuestos"}\NormalTok{,}
    \StringTok{"Taxes"} \OtherTok{=} \StringTok{"Impuestos"}\NormalTok{,}
    \StringTok{"Ordinary\_Activities\_Profit"} \OtherTok{=} \StringTok{"Rdo..Actividades.Odinarias"}\NormalTok{,}
    \StringTok{"Extraordinary\_and\_Other\_Profit"} \OtherTok{=} \StringTok{"Rdo..Extr..y.Otros"}\NormalTok{,}
    \StringTok{"Net\_Profit"} \OtherTok{=} \StringTok{"Rdo.Ejercicio"}\NormalTok{,}
    \StringTok{"ROE"} \OtherTok{=} \StringTok{"ROE"}\NormalTok{,}
    \StringTok{"ROA"} \OtherTok{=} \StringTok{"ROA"}\NormalTok{,}
    \StringTok{"Collection\_Period"} \OtherTok{=} \StringTok{"Período.de.Cobro"}\NormalTok{,}
    \StringTok{"Credit\_Period"} \OtherTok{=} \StringTok{"Período.de.Credito"}\NormalTok{,}
    \StringTok{"ROEE"} \OtherTok{=} \StringTok{"ROEE"}\NormalTok{,}
    \StringTok{"ROAA"} \OtherTok{=} \StringTok{"ROAA"}\NormalTok{,}
    \StringTok{"CollectionPeriod"} \OtherTok{=} \StringTok{"PeríodoCobro"}\NormalTok{,}
    \StringTok{"Payment\_Period"} \OtherTok{=} \StringTok{"PeríodoPago"}\NormalTok{,}
    \StringTok{"PMC\_PMP"} \OtherTok{=} \StringTok{"PMC{-}PMP"}\NormalTok{,}
    \StringTok{"Net\_Asset\_Turnover"} \OtherTok{=} \StringTok{"Rotación.de.activos.netos"}\NormalTok{,}
    \StringTok{"Inventory\_Turnover"} \OtherTok{=} \StringTok{"Rotación.de.las.existencias"}\NormalTok{,}
    \StringTok{"Solvency\_Turnover"} \OtherTok{=} \StringTok{"Rotacion.de.Solvencia"}\NormalTok{,}
    \StringTok{"Asset\_Turnover"} \OtherTok{=} \StringTok{"RotacActivos"}\NormalTok{,}
    \StringTok{"InventoryTurnover"} \OtherTok{=} \StringTok{"RotacExistenc"}\NormalTok{,}
    \StringTok{"SolvencyTurnover"} \OtherTok{=} \StringTok{"RotacSolvencia"}\NormalTok{,}
    \StringTok{"Liquidity\_Ratio"} \OtherTok{=} \StringTok{"Ratio.de.Liquidez"}\NormalTok{,}
    \StringTok{"Leverage"} \OtherTok{=} \StringTok{"Apalancamiento"}\NormalTok{,}
    \StringTok{"Profit\_per\_Employee"} \OtherTok{=} \StringTok{"Beneficio.por.empleado"}\NormalTok{,}
    \StringTok{"Operating\_Revenue\_per\_Employee"} \OtherTok{=} \StringTok{"Ingresos.Explotación.por.empleado"}\NormalTok{,}
    \StringTok{"Average\_Employee\_Cost"} \OtherTok{=} \StringTok{"Coste.medio.Empleados"}\NormalTok{,}
    \StringTok{"Total\_Assets\_per\_Employee"} \OtherTok{=} \StringTok{"Total.acivos.por.empleado"}\NormalTok{,}
    \StringTok{"Levera"} \OtherTok{=} \StringTok{"Apalancam"}\NormalTok{,}
    \StringTok{"Profit\_Employee"} \OtherTok{=} \StringTok{"Benefic/empleado"}\NormalTok{,}
    \StringTok{"OperatingRevenue\_Employee"} \OtherTok{=} \StringTok{"IngrExpl/empleado"}\NormalTok{,}
    \StringTok{"Cost\_Employee"} \OtherTok{=} \StringTok{"Coste/empleado"}\NormalTok{,}
    \StringTok{"Assets\_Employee"} \OtherTok{=} \StringTok{"Activos/empleado"}\NormalTok{,}
    \StringTok{"Number\_of\_Board\_and\_Management\_Members"} \OtherTok{=} 
      \StringTok{"Nümero.de.miembros.de.las.juntas.\&.gestión"}\NormalTok{,}
    \StringTok{"Board\_Members"} \OtherTok{=} \StringTok{"MiembrosJuntas"}\NormalTok{,}
    \StringTok{"DM\_Full\_Name"} \OtherTok{=} \StringTok{"DM.Nombre.completo"}\NormalTok{,}
    \StringTok{"DM\_Job\_Title"} \OtherTok{=} \StringTok{"DM.Título.trabajo"}\NormalTok{,}
    \StringTok{"Shareholder\_Direct\_Percentage"} \OtherTok{=} \StringTok{"Accionista.{-}.\%.directo"}\NormalTok{,}
    \StringTok{"Shareholder\_Total\_Percentage"} \OtherTok{=} \StringTok{"Accionista.{-}.\%.total"}\NormalTok{,}
    \StringTok{"CSH\_Direct\_Percentage"} \OtherTok{=} \StringTok{"CSH.{-}.\%.directo"}\NormalTok{,}
    \StringTok{"DM\_Original\_Job\_Title"} \OtherTok{=} \StringTok{"DM.Título.original.trabajo"}\NormalTok{,}
    \StringTok{"DM\_Board\_Committee\_or\_Executive\_Department"} \OtherTok{=} 
      \StringTok{"DM.Junta,.comité.or.departamento.ejecutivo"}\NormalTok{,}
    \StringTok{"DM\_Level\_of\_Responsibility"} \OtherTok{=} \StringTok{"DM.Nivel.de.responsabilidad"}\NormalTok{,}
    \StringTok{"DM\_First\_Name"} \OtherTok{=} \StringTok{"DM.Nombre"}\NormalTok{,}
    \StringTok{"DM\_Last\_Name"} \OtherTok{=} \StringTok{"DM.Apellido"}\NormalTok{,}
    \StringTok{"DM\_Gender"} \OtherTok{=} \StringTok{"DM.Género"}\NormalTok{,}
    \StringTok{"DM\_Nationality\_Country"} \OtherTok{=} \StringTok{"DM.País.de.nacionalidad"}\NormalTok{,}
    \StringTok{"DM\_Also\_a\_Shareholder"} \OtherTok{=} \StringTok{"DM.También.un.accionista"}\NormalTok{,}
    \StringTok{"DM\_Position\_Type"} \OtherTok{=} \StringTok{"DM.Tipo.de.posición"}\NormalTok{,}
    \StringTok{"Number\_of\_Advisors"} \OtherTok{=} \StringTok{"Número.de.asesores"}\NormalTok{,}
    \StringTok{"ADV\_First\_Name"} \OtherTok{=} \StringTok{"ADV.Nombre"}\NormalTok{,}
    \StringTok{"ADV\_Last\_Name"} \OtherTok{=} \StringTok{"ADV.Apellido"}\NormalTok{,}
    \StringTok{"ADV\_Gender"} \OtherTok{=} \StringTok{"ADV.Género"}\NormalTok{,}
    \StringTok{"ADV\_Nationality\_Country"} \OtherTok{=} \StringTok{"ADV.País.de.nacionalidad"}\NormalTok{,}
    \StringTok{"Nationality\_Country"} \OtherTok{=} \StringTok{"País.de.nacionalidad"}\NormalTok{,}
    \StringTok{"Number\_of\_Employees"} \OtherTok{=} \StringTok{"Número.empleados"}\NormalTok{,}
    \StringTok{"BvD\_Independence\_Indicator"} \OtherTok{=} \StringTok{"Indicador.independencia.BvD"}
\NormalTok{  )}
\end{Highlighting}
\end{Shaded}

\section{Coercion de datos}\label{coercion-de-datos}

Las funciones de coerción en R son fundamentales para transformar los
tipos de datos y garantizar que sean compatibles con los análisis o
manipulaciones que se desean realizar. En este caso, se emplea la
función parse\_number del paquete readr, que es especialmente útil para
convertir valores almacenados como texto en datos numéricos. Esta
función extrae automáticamente los componentes numéricos de una cadena
de texto y los convierte en un tipo de dato numérico, parse\_number
eliminará los caracteres no numéricos, como letras, símbolos o comas
Este proceso es clave para trabajar con columnas que combinan datos
numéricos y caracteres en un mismo campo, ya que permite su correcta
utilización en operaciones matemáticas y estadísticas.

Adicionalmente, se utiliza la función mutate del paquete dplyr, la cual
es útil para modificar o crear nuevas columnas en un conjunto de datos.
Dentro de mutate, se aplica parse\_number a las columnas seleccionadas
para asegurar su transformación. Esto permite trabajar de forma
eficiente con múltiples variables en un solo paso, sin necesidad de
manipularlas de forma individual.

\begin{Shaded}
\begin{Highlighting}[]
\FunctionTok{colnames}\NormalTok{(DATA)}
\FunctionTok{str}\NormalTok{(DATA)}
\NormalTok{DATA\_Manipulada }\OtherTok{\textless{}{-}}\NormalTok{ DATA }\SpecialCharTok{\%\textgreater{}\%}
  \FunctionTok{mutate}\NormalTok{(}
    \CommentTok{\# Conversión de columnas a numérico}
    \AttributeTok{Growth =} \FunctionTok{parse\_number}\NormalTok{(Growth, }
                          \AttributeTok{locale =} \FunctionTok{locale}\NormalTok{(}\AttributeTok{decimal\_mark =} \StringTok{"."}\NormalTok{)),}
    \AttributeTok{Ln\_Inflation =} \FunctionTok{parse\_number}\NormalTok{(Ln\_Inflation, }
                                \AttributeTok{locale =} \FunctionTok{locale}\NormalTok{(}\AttributeTok{decimal\_mark =} \StringTok{"."}\NormalTok{)),}
    \AttributeTok{Ln\_Seniority =} \FunctionTok{parse\_number}\NormalTok{(Ln\_Seniority, }
                                \AttributeTok{locale =} \FunctionTok{locale}\NormalTok{(}\AttributeTok{decimal\_mark =} \StringTok{"."}\NormalTok{)),}
    \AttributeTok{Ln\_Fixed\_Assets =} \FunctionTok{parse\_number}\NormalTok{(Ln\_Fixed\_Assets, }
                                   \AttributeTok{locale =} \FunctionTok{locale}\NormalTok{(}\AttributeTok{decimal\_mark =} \StringTok{"."}\NormalTok{)),}
    \AttributeTok{Ln\_Current\_Assets =} \FunctionTok{parse\_number}\NormalTok{(Ln\_Current\_Assets, }
                                     \AttributeTok{locale =} \FunctionTok{locale}\NormalTok{(}\AttributeTok{decimal\_mark =} \StringTok{"."}\NormalTok{)),}
    \AttributeTok{Ln\_Inventory =} \FunctionTok{parse\_number}\NormalTok{(Ln\_Inventory, }
                                \AttributeTok{locale =} \FunctionTok{locale}\NormalTok{(}\AttributeTok{decimal\_mark =} \StringTok{"."}\NormalTok{)),}
    \AttributeTok{Ln\_Receivables =} \FunctionTok{parse\_number}\NormalTok{(Ln\_Receivables, }
                                  \AttributeTok{locale =} \FunctionTok{locale}\NormalTok{(}\AttributeTok{decimal\_mark =} \StringTok{"."}\NormalTok{)),}
    \AttributeTok{Ln\_Other\_Assets =} \FunctionTok{parse\_number}\NormalTok{(Ln\_Other\_Assets, }
                                   \AttributeTok{locale =} \FunctionTok{locale}\NormalTok{(}\AttributeTok{decimal\_mark =} \StringTok{"."}\NormalTok{)),}
    \AttributeTok{Ln\_Cash =} \FunctionTok{parse\_number}\NormalTok{(Ln\_Cash, }\AttributeTok{locale =} 
                             \FunctionTok{locale}\NormalTok{(}\AttributeTok{decimal\_mark =} \StringTok{"."}\NormalTok{)),}
    \AttributeTok{Ln\_Total\_Assets =} \FunctionTok{parse\_number}\NormalTok{(Ln\_Total\_Assets, }
                                   \AttributeTok{locale =} \FunctionTok{locale}\NormalTok{(}\AttributeTok{decimal\_mark =} \StringTok{"."}\NormalTok{)),}
    \AttributeTok{Liquidity1 =} \FunctionTok{parse\_number}\NormalTok{(Liquidity1, }
                              \AttributeTok{locale =} \FunctionTok{locale}\NormalTok{(}\AttributeTok{decimal\_mark =} \StringTok{"."}\NormalTok{)),}
    \AttributeTok{Liquidity1\_Dummy =} \FunctionTok{parse\_number}\NormalTok{(Liquidity1\_Dummy, }
                                    \AttributeTok{locale =} \FunctionTok{locale}\NormalTok{(}\AttributeTok{decimal\_mark =} \StringTok{"."}\NormalTok{)),}
    \AttributeTok{Ln\_Non\_Current\_Liabilities =}\NormalTok{ parse\_number}
\NormalTok{    (Ln\_Non\_Current\_Liabilities, }
      \AttributeTok{locale =} \FunctionTok{locale}\NormalTok{(}\AttributeTok{decimal\_mark =} \StringTok{"."}\NormalTok{)),}
    \AttributeTok{Ln\_Current\_Liabilities =} \FunctionTok{parse\_number}\NormalTok{(Ln\_Current\_Liabilities, }
                                          \AttributeTok{locale =} \FunctionTok{locale}\NormalTok{(}\AttributeTok{decimal\_mark =} \StringTok{"."}\NormalTok{)),}
    \AttributeTok{Ln\_Total\_Liabilities =} \FunctionTok{parse\_number}\NormalTok{(Ln\_Total\_Liabilities, }
                                        \AttributeTok{locale =} \FunctionTok{locale}\NormalTok{(}\AttributeTok{decimal\_mark =} \StringTok{"."}\NormalTok{)),}
    \AttributeTok{Ln\_Equity =} \FunctionTok{parse\_number}\NormalTok{(Ln\_Equity, }
                             \AttributeTok{locale =} \FunctionTok{locale}\NormalTok{(}\AttributeTok{decimal\_mark =} \StringTok{"."}\NormalTok{)),}
    \AttributeTok{ROE =} \FunctionTok{parse\_number}\NormalTok{(ROE, }\AttributeTok{locale =} 
                         \FunctionTok{locale}\NormalTok{(}\AttributeTok{decimal\_mark =} \StringTok{"."}\NormalTok{)),}
    \AttributeTok{ROA =} \FunctionTok{parse\_number}\NormalTok{(ROA, }
                       \AttributeTok{locale =} \FunctionTok{locale}\NormalTok{(}\AttributeTok{decimal\_mark =} \StringTok{"."}\NormalTok{)),}
    \AttributeTok{Collection\_Period =} \FunctionTok{parse\_number}\NormalTok{(Collection\_Period, }
                                     \AttributeTok{locale =} \FunctionTok{locale}\NormalTok{(}\AttributeTok{decimal\_mark =} \StringTok{"."}\NormalTok{)),}
    \AttributeTok{Credit\_Period =} \FunctionTok{parse\_number}\NormalTok{(Credit\_Period, }
                                 \AttributeTok{locale =} \FunctionTok{locale}\NormalTok{(}\AttributeTok{decimal\_mark =} \StringTok{"."}\NormalTok{)),}
    \AttributeTok{ROEE =} \FunctionTok{parse\_number}\NormalTok{(ROEE, }\AttributeTok{locale =} 
                          \FunctionTok{locale}\NormalTok{(}\AttributeTok{decimal\_mark =} \StringTok{"."}\NormalTok{)),}
    \AttributeTok{ROAA =} \FunctionTok{parse\_number}\NormalTok{(ROAA, }
                        \AttributeTok{locale =} \FunctionTok{locale}\NormalTok{(}\AttributeTok{decimal\_mark =} \StringTok{"."}\NormalTok{)),}
    \AttributeTok{CollectionPeriod =} \FunctionTok{parse\_number}\NormalTok{(CollectionPeriod, }
                                    \AttributeTok{locale =} \FunctionTok{locale}\NormalTok{(}\AttributeTok{decimal\_mark =} \StringTok{"."}\NormalTok{)),}
    \AttributeTok{Payment\_Period =} \FunctionTok{parse\_number}\NormalTok{(Payment\_Period, }
                                  \AttributeTok{locale =} \FunctionTok{locale}\NormalTok{(}\AttributeTok{decimal\_mark =} \StringTok{"."}\NormalTok{)),}
    \AttributeTok{PMC\_PMP =} \FunctionTok{parse\_number}\NormalTok{(PMC\_PMP, }
                           \AttributeTok{locale =} \FunctionTok{locale}\NormalTok{(}\AttributeTok{decimal\_mark =} \StringTok{"."}\NormalTok{)),}
    \AttributeTok{Net\_Asset\_Turnover =} \FunctionTok{parse\_number}\NormalTok{(Net\_Asset\_Turnover, }
                                      \AttributeTok{locale =} \FunctionTok{locale}\NormalTok{(}\AttributeTok{decimal\_mark =} \StringTok{"."}\NormalTok{)),}
    \AttributeTok{Inventory\_Turnover =} \FunctionTok{parse\_number}\NormalTok{(Inventory\_Turnover, }
                                      \AttributeTok{locale =} \FunctionTok{locale}\NormalTok{(}\AttributeTok{decimal\_mark =} \StringTok{"."}\NormalTok{)),}
    \AttributeTok{Solvency\_Turnover =} \FunctionTok{parse\_number}\NormalTok{(Solvency\_Turnover, }
                                     \AttributeTok{locale =} \FunctionTok{locale}\NormalTok{(}\AttributeTok{decimal\_mark =} \StringTok{"."}\NormalTok{)),}
    \AttributeTok{Asset\_Turnover =} \FunctionTok{parse\_number}\NormalTok{(Asset\_Turnover, }
                                  \AttributeTok{locale =} \FunctionTok{locale}\NormalTok{(}\AttributeTok{decimal\_mark =} \StringTok{"."}\NormalTok{)),}
    \AttributeTok{InventoryTurnover =} \FunctionTok{parse\_number}\NormalTok{(InventoryTurnover, }
                                     \AttributeTok{locale =} \FunctionTok{locale}\NormalTok{(}\AttributeTok{decimal\_mark =} \StringTok{"."}\NormalTok{)),}
    \AttributeTok{SolvencyTurnover =} \FunctionTok{parse\_number}\NormalTok{(SolvencyTurnover, }
                                    \AttributeTok{locale =} \FunctionTok{locale}\NormalTok{(}\AttributeTok{decimal\_mark =} \StringTok{"."}\NormalTok{)),}
    \AttributeTok{Liquidity\_Ratio =} \FunctionTok{parse\_number}\NormalTok{(Liquidity\_Ratio, }
                                   \AttributeTok{locale =} \FunctionTok{locale}\NormalTok{(}\AttributeTok{decimal\_mark =} \StringTok{"."}\NormalTok{)),}
    \AttributeTok{Leverage =} \FunctionTok{parse\_number}\NormalTok{(Leverage, }\AttributeTok{locale =} 
                              \FunctionTok{locale}\NormalTok{(}\AttributeTok{decimal\_mark =} \StringTok{"."}\NormalTok{)),}
    \AttributeTok{Profit\_per\_Employee =} \FunctionTok{parse\_number}\NormalTok{(Profit\_per\_Employee, }
                                       \AttributeTok{locale =} \FunctionTok{locale}\NormalTok{(}\AttributeTok{decimal\_mark =} \StringTok{"."}\NormalTok{)),}
    \AttributeTok{Operating\_Revenue\_per\_Employee =}\NormalTok{ parse\_number}
\NormalTok{    (Operating\_Revenue\_per\_Employee, }\AttributeTok{locale =} \FunctionTok{locale}\NormalTok{(}\AttributeTok{decimal\_mark =} \StringTok{"."}\NormalTok{)),}
    \AttributeTok{Levera =} \FunctionTok{parse\_number}\NormalTok{(Levera, }
                          \AttributeTok{locale =} \FunctionTok{locale}\NormalTok{(}\AttributeTok{decimal\_mark =} \StringTok{"."}\NormalTok{)),}
    \AttributeTok{Profit\_Employee =} \FunctionTok{parse\_number}\NormalTok{(Profit\_Employee, }
                                   \AttributeTok{locale =} \FunctionTok{locale}\NormalTok{(}\AttributeTok{decimal\_mark =} \StringTok{"."}\NormalTok{)),}
    \AttributeTok{OperatingRevenue\_Employee =}\NormalTok{ parse\_number}
\NormalTok{    (OperatingRevenue\_Employee, }\AttributeTok{locale =} \FunctionTok{locale}\NormalTok{(}\AttributeTok{decimal\_mark =} \StringTok{"."}\NormalTok{)),}
    \AttributeTok{Shareholder\_Direct\_Percentage =}\NormalTok{ parse\_number}
\NormalTok{    (Shareholder\_Direct\_Percentage, }\AttributeTok{locale =} \FunctionTok{locale}\NormalTok{(}\AttributeTok{decimal\_mark =} \StringTok{"."}\NormalTok{)),}
    \AttributeTok{Shareholder\_Total\_Percentage =}\NormalTok{ parse\_number}
\NormalTok{    (Shareholder\_Total\_Percentage, }\AttributeTok{locale =} \FunctionTok{locale}\NormalTok{(}\AttributeTok{decimal\_mark =} \StringTok{"."}\NormalTok{)),}
    \AttributeTok{CSH\_Direct\_Percentage =} \FunctionTok{parse\_number}\NormalTok{(CSH\_Direct\_Percentage, }
                                         \AttributeTok{locale =} \FunctionTok{locale}\NormalTok{(}\AttributeTok{decimal\_mark =} \StringTok{"."}\NormalTok{)),}
 
    \CommentTok{\# Conversión de columnas a Date }
    \AttributeTok{Incorporation\_Date =} \FunctionTok{as.Date}\NormalTok{(Incorporation\_Date, }\AttributeTok{origin =} \StringTok{"1899{-}12{-}30"}\NormalTok{),}
    \AttributeTok{End\_Date =} \FunctionTok{as.Date}\NormalTok{(End\_Date, }\AttributeTok{origin =} \StringTok{"1899{-}12{-}30"}\NormalTok{),}

   \CommentTok{\# Conversión de columnas a character}
    \AttributeTok{Country =} \FunctionTok{as.character}\NormalTok{(Country),}
    \AttributeTok{Identity =} \FunctionTok{as.character}\NormalTok{(Identity),}
    \AttributeTok{Legal\_Form =} \FunctionTok{as.character}\NormalTok{(Legal\_Form),}
    \AttributeTok{Legal\_Form\_Tabul =} \FunctionTok{as.character}\NormalTok{(Legal\_Form\_Tabul),}
   
   \CommentTok{\#Conversion de columnas a factor}
    \AttributeTok{ADV\_Gender=} \FunctionTok{parse\_factor}\NormalTok{(ADV\_Gender,}
                             \AttributeTok{levels =} \FunctionTok{c}\NormalTok{(}\StringTok{"M"}\NormalTok{,}\StringTok{"F"}\NormalTok{,}\StringTok{"M}\SpecialCharTok{\textbackslash{}n}\StringTok{M"}\NormalTok{,}\StringTok{"M}\SpecialCharTok{\textbackslash{}n}\StringTok{M}\SpecialCharTok{\textbackslash{}n}\StringTok{M"}\NormalTok{,}\StringTok{"M}\SpecialCharTok{\textbackslash{}n}\StringTok{F"}\NormalTok{),}
                             \AttributeTok{ordered =} \ConstantTok{TRUE}\NormalTok{),}
    \AttributeTok{BvD\_Independence\_Indicator=} \FunctionTok{parse\_factor}\NormalTok{(BvD\_Independence\_Indicator,}
                                             \AttributeTok{levels =} \FunctionTok{c}\NormalTok{(}\StringTok{"A+"}\NormalTok{, }\StringTok{"A"}\NormalTok{, }\StringTok{"A{-}"}\NormalTok{, }
                                                        \StringTok{"B+"}\NormalTok{, }\StringTok{"B"}\NormalTok{, }\StringTok{"B{-}"}\NormalTok{, }\StringTok{"C+"}\NormalTok{,}
                                                        \StringTok{"C"}\NormalTok{, }\StringTok{"C{-}"}\NormalTok{, }\StringTok{"D"}\NormalTok{, }\StringTok{"U"}\NormalTok{),}
                                             \AttributeTok{ordered =} \ConstantTok{TRUE}\NormalTok{),}
    \AttributeTok{Standard\_Legal\_Form =} \FunctionTok{parse\_factor}\NormalTok{(Standard\_Legal\_Form,}
                                       \AttributeTok{levels =} \FunctionTok{c}\NormalTok{(}\StringTok{"Public limited companies"}\NormalTok{, }
                                                  \StringTok{"Private limited companies"}\NormalTok{, }
                                                  \StringTok{"Partnerships"}\NormalTok{, }
                                                  \StringTok{"Other legal forms"}\NormalTok{),}
                                       \AttributeTok{ordered =} \ConstantTok{FALSE}\NormalTok{))}
\end{Highlighting}
\end{Shaded}

\section{Descripción de las
variables}\label{descripciuxf3n-de-las-variables}

\subsection{Tabla 1}\label{tabla-1}

Esta investigación se ha centrado en examinar los factores que
determinan, en mayor medida, la rentabilidad de las empresas que operan
en el sector de piedras naturales. Para el estudio del rendimiento de
las empresas, las variables más comúnmente utilizadas son ROA
(rentabilidad sobre activos---rentabilidad económica) y ROE
(rentabilidad sobre el capital---rentabilidad financiera), debido a su
capacidad para medir las inversiones en términos de activos y
patrimonio. Investigaciones previas han utilizado ROA y ROE como
variables dependientes. Por lo tanto, en los dos análisis empíricos
implementados en este estudio, ROA y ROE fueron consideradas como
variables dependientes. Ambas son variables cuantitativas continuas. Las
variables explicativas bajo análisis pueden agruparse en tres categorías
distintas:

\begin{itemize}
\item
  Variables asociadas a la empresa, como la edad de la empresa en el
  mercado, así como variables financieras como el volumen de activos,
  apalancamiento, ingreso operativo total, rotación de existencias,
  rotación de activos, periodo promedio de cobro, periodo promedio de
  pago, crecimiento de la empresa, y la forma jurídica.
\item
  Variables asociadas al entorno económico, como el país, el producto
  interno bruto (PIB), y el nivel de inflación.
\item
  Variables vinculadas a la diversidad en la gestión empresarial,
  identificada en esta investigación por la variable de género.
\end{itemize}

Las siguientes variables independientes fueron consideradas en el
estudio: tamaño de la empresa, medido por el volumen de activos; el
grado de apalancamiento; el crecimiento de la empresa; el cambio en el
PIB del país; la inflación; y el porcentaje de directores de la junta
femenina. Finalmente, las siguientes variables fueron utilizadas como
variables de control: ingreso operativo, rotación de existencias,
rotación de activos, periodos promedio de cobro y pago, edad de la
empresa, y forma jurídica.

\begin{verbatim}
Rentabilidad financiera (ROE)
\end{verbatim}

El indicador ROE expresa la capacidad de una empresa para generar
ganancias a través de un uso productivo de las contribuciones de los
accionistas y una gestión eficiente. Se calcula como la relación entre
el beneficio neto después de impuestos y el patrimonio de los
accionistas. 3.2.2. Rentabilidad económica (ROA) ROA se define como el
beneficio neto después de impuestos dividido por el total de los
activos, y también ha sido ampliamente utilizado en la literatura.

\begin{verbatim}
Tamaño de la empresa
\end{verbatim}

El tamaño de la empresa es considerado un factor importante al explicar
la rentabilidad. La teoría sugiere que las empresas más grandes tienen
mayor acceso a los mercados financieros y obtienen mejores tasas de
interés al explotar las economías de escala. Según algunos estudios, el
tamaño de la empresa tiene un efecto positivo y significativo sobre la
rentabilidad.

\begin{verbatim}
Endeudamiento
\end{verbatim}

El endeudamiento es uno de los factores más críticos al analizar el
rendimiento corporativo. El ratio de deuda se define como el total de
deudas de una empresa dividido por sus activos totales. Los resultados
de investigaciones anteriores generalmente encuentran una relación
inversa entre el nivel de endeudamiento de una empresa y su
rentabilidad.

\begin{verbatim}
Crecimiento
\end{verbatim}

Algunas investigaciones previas utilizaron el crecimiento de la empresa
como el cambio porcentual en los ingresos operativos. Sin embargo, como
en este estudio, otros investigadores consideran el crecimiento como el
cambio porcentual en los activos totales, mostrando una relación
positiva y significativa entre el cambio porcentual en los activos y la
rentabilidad de la empresa.

\begin{verbatim}
Variación del PIB
\end{verbatim}

El PIB es uno de los indicadores más utilizados para medir la actividad
económica dentro de un país. El crecimiento económico refleja las
condiciones macroeconómicas generales. Se presume que un cambio en el
PIB puede influir en el rendimiento de las empresas, con una relación
positiva esperada entre la variación del PIB y la rentabilidad.

\begin{verbatim}
Inflación
\end{verbatim}

El índice de inflación es otro indicador macroeconómico comúnmente
utilizado en estudios de rentabilidad. El efecto de la inflación en la
rentabilidad de una empresa depende de si la inflación es anticipada o
no. Se espera que haya una relación negativa entre la inflación no
anticipada y la rentabilidad de las empresas.

\begin{verbatim}
Género
\end{verbatim}

Muchos estudios han analizado la influencia de la diversidad de género
en las juntas directivas sobre la rentabilidad de las empresas,
encontrando que una mayor presencia femenina tiene una influencia
positiva en la rentabilidad.

\begin{verbatim}
Ingreso operativo
\end{verbatim}

El ingreso operativo de una empresa es considerado un indicador clave de
muchos aspectos positivos que apoyan tanto el crecimiento como la
rentabilidad. Este estudio utiliza el logaritmo natural del ingreso
operativo para determinar su relación con la rentabilidad de la empresa.

\begin{verbatim}
Rotación de existencias
\end{verbatim}

El ratio de rotación de existencias es una medida importante para
evaluar la eficiencia de la gestión de inventarios. Una alta rotación de
existencias generalmente indica una gestión eficiente. Se espera que
esta relación sea positiva con la rentabilidad.

\begin{verbatim}
Rotación de activos
\end{verbatim}

La rotación de activos mide la eficiencia de una empresa en el uso de
sus activos para generar ingresos operativos. Los resultados previos
muestran resultados mixtos, con algunos estudios que indican que la
rotación de activos tiene un impacto positivo en la rentabilidad.

\begin{verbatim}
Periodo promedio de cobro
\end{verbatim}

El periodo promedio de cobro representa el número promedio de días que
una empresa tarda en cobrar después de una venta a crédito. En algunos
estudios, se encuentra que existe una relación positiva significativa
entre el periodo promedio de cobro y la rentabilidad, mientras que otros
autores encuentran una relación negativa.

\begin{verbatim}
Periodo promedio de pago
\end{verbatim}

Este indicador muestra el número promedio de días que una empresa tarda
en pagar sus deudas a corto plazo. Estudios empíricos han encontrado una
relación negativa entre el periodo promedio de pago y la rentabilidad de
la empresa.

\begin{verbatim}
Edad
\end{verbatim}

La edad de la empresa, definida como el número de años que ha estado en
el mercado, tiene una relación compleja con la rentabilidad. Mientras
que algunos estudios encuentran una relación positiva significativa,
otros argumentan que la edad tiene un efecto negativo sobre la
rentabilidad.

\begin{verbatim}
Forma jurídica
\end{verbatim}

La forma jurídica ha sido utilizada para analizar la rentabilidad,
particularmente en empresas de responsabilidad limitada. Sin embargo,
algunos estudios no han encontrado una relación significativa entre la
forma jurídica y la rentabilidad.

\begin{verbatim}
País
\end{verbatim}

El país en el que se ubica la empresa puede influir significativamente
en su rentabilidad, ya que las condiciones económicas y regulatorias
varían según la región.

\begin{Shaded}
\begin{Highlighting}[]
\FunctionTok{library}\NormalTok{(tibble)}

\CommentTok{\# Crear la tabla de variables con saltos de línea}
\NormalTok{tabla\_esp }\OtherTok{\textless{}{-}}\NormalTok{ tibble}\SpecialCharTok{::}\FunctionTok{tibble}\NormalTok{(}
  \AttributeTok{Abreviatura =} \FunctionTok{c}\NormalTok{(}\StringTok{"ROE"}\NormalTok{, }\StringTok{"ROA"}\NormalTok{, }\StringTok{"Size"}\NormalTok{, }\StringTok{"Debt"}\NormalTok{, }\StringTok{"Growth"}\NormalTok{, }\StringTok{"VarGDP"}\NormalTok{, }\StringTok{"Inflat"}\NormalTok{, }
                   \StringTok{"Gender"}\NormalTok{, }\StringTok{"OpInc"}\NormalTok{, }\StringTok{"StockT"}\NormalTok{, }\StringTok{"AssetT"}\NormalTok{, }\StringTok{"ARP"}\NormalTok{, }\StringTok{"APP"}\NormalTok{, }
                   \StringTok{"Age"}\NormalTok{, }\StringTok{"LForm"}\NormalTok{, }\StringTok{"Country"}\NormalTok{),}
  \AttributeTok{Variable =} \FunctionTok{c}\NormalTok{(}\StringTok{"Retorno sobre el patrimonio"}\NormalTok{, }\StringTok{"Retorno sobre los activos"}\NormalTok{,}
               \StringTok{"Tamaño de la empresa"}\NormalTok{, }\StringTok{"Endeudamiento"}\NormalTok{, }
               \StringTok{"Crecimiento de la empresa"}\NormalTok{, }\StringTok{"Cambio en el PIB"}\NormalTok{, }
               \StringTok{"Inflación del país"}\NormalTok{, }\StringTok{"Diversidad de género"}\NormalTok{,}
               \StringTok{"Ingreso operativo"}\NormalTok{, }\StringTok{"Rotación del inventario"}\NormalTok{,}
               \StringTok{"Rotación de activos"}\NormalTok{, }\StringTok{"Período promedio de cobro"}\NormalTok{,}
               \StringTok{"Período promedio de pago"}\NormalTok{, }\StringTok{"Edad de la empresa"}\NormalTok{,}
               \StringTok{"Forma jurídica"}\NormalTok{, }\StringTok{"País de residencia de la empresa"}\NormalTok{),}
\NormalTok{  Definición }\OtherTok{=} \FunctionTok{c}\NormalTok{(}
    \StringTok{"Beneficio neto dividido por el patrimonio"}\NormalTok{, }
    \StringTok{"Beneficio neto dividido por los activos totales"}\NormalTok{, }
    \StringTok{"Logaritmo natural del total de activos de la empresa"}\NormalTok{, }
    \StringTok{"Pasivos totales divididos por los activos totales"}\NormalTok{, }
    \StringTok{"Porcentaje de cambio en los activos totales"}\NormalTok{, }
    \StringTok{"Porcentaje de cambio en el PIB"}\NormalTok{, }
    \StringTok{"Inflación del país en el año"}\NormalTok{, }
    \StringTok{"Porcentaje de mujeres en el consejo de administración"}\NormalTok{, }
    \StringTok{"Logaritmo natural del ingreso operativo"}\NormalTok{, }
    \StringTok{"Costo de ventas dividido por inventario"}\NormalTok{, }
    \StringTok{"Ingreso operativo dividido por activos totales"}\NormalTok{, }
    \StringTok{"Promedio de días que la empresa tarda en recibir pagos"}\NormalTok{, }
    \StringTok{"Promedio de días que la empresa tarda en pagar a proveedores"}\NormalTok{, }
    \StringTok{"Edad de la empresa en años"}\NormalTok{, }
    \StringTok{"La forma jurídica de la empresa: Sociedad anónima, }
\StringTok{    Sociedad limitada, Cooperativa, Otras formas legales"}\NormalTok{,}
    \StringTok{"Variable igual a 1 para España y 0 para Italia"}
\NormalTok{  )}
\NormalTok{)}
\end{Highlighting}
\end{Shaded}

\begin{Shaded}
\begin{Highlighting}[]
\CommentTok{\# Crear la tabla en gt()}
\NormalTok{tabla\_esp }\SpecialCharTok{\%\textgreater{}\%}
  \FunctionTok{gt}\NormalTok{() }\SpecialCharTok{\%\textgreater{}\%}
  \FunctionTok{tab\_header}\NormalTok{(}
    \AttributeTok{title =} \FunctionTok{md}\NormalTok{(}\StringTok{"**Tabla 1. Definición de variables (Español)**"}\NormalTok{)}
\NormalTok{  ) }\SpecialCharTok{\%\textgreater{}\%}
  \FunctionTok{cols\_label}\NormalTok{(}
    \AttributeTok{Abreviatura =} \FunctionTok{md}\NormalTok{(}\StringTok{"**Abreviatura**"}\NormalTok{),}
    \AttributeTok{Variable =} \FunctionTok{md}\NormalTok{(}\StringTok{"**Variable**"}\NormalTok{),}
\NormalTok{    Definición }\OtherTok{=} \FunctionTok{md}\NormalTok{(}\StringTok{"**Definición**"}\NormalTok{)}
\NormalTok{  ) }\SpecialCharTok{\%\textgreater{}\%}
  \FunctionTok{cols\_width}\NormalTok{(}
\NormalTok{    Definición }\SpecialCharTok{\textasciitilde{}} \FunctionTok{px}\NormalTok{(}\DecValTok{300}\NormalTok{)  }\CommentTok{\# Ajusta el ancho de la columna Definición}
\NormalTok{  ) }\SpecialCharTok{\%\textgreater{}\%}
  \FunctionTok{tab\_options}\NormalTok{(}
    \AttributeTok{column\_labels.font.size =} \FunctionTok{px}\NormalTok{(}\DecValTok{12}\NormalTok{),}
    \AttributeTok{table.font.size =} \FunctionTok{px}\NormalTok{(}\DecValTok{12}\NormalTok{),}
    \AttributeTok{data\_row.padding =} \FunctionTok{px}\NormalTok{(}\DecValTok{5}\NormalTok{),}
    \AttributeTok{row.striping.include\_table\_body =} \ConstantTok{TRUE}
\NormalTok{  ) }\SpecialCharTok{\%\textgreater{}\%}
  \FunctionTok{tab\_style}\NormalTok{(}
    \AttributeTok{style =} \FunctionTok{list}\NormalTok{(}
      \FunctionTok{cell\_text}\NormalTok{(}\AttributeTok{align =} \StringTok{"center"}\NormalTok{)  }
\NormalTok{    ),}
    \AttributeTok{locations =} \FunctionTok{cells\_title}\NormalTok{(}\AttributeTok{groups =} \StringTok{"title"}\NormalTok{)  }
\NormalTok{  )}
\end{Highlighting}
\end{Shaded}

\begin{table}[!t]
\caption*{
{\large \textbf{Tabla 1. Definición de variables (Español)}}
} 
\fontsize{9.0pt}{10.8pt}\selectfont
\begin{tabular*}{\linewidth}{@{\extracolsep{\fill}}ll>{\raggedright\arraybackslash}p{\dimexpr 225.00pt -2\tabcolsep-1.5\arrayrulewidth}}
\toprule
\textbf{Abreviatura} & \textbf{Variable} & \textbf{Definición} \\ 
\midrule\addlinespace[2.5pt]
ROE & Retorno sobre el patrimonio & Beneficio neto dividido por el patrimonio \\ 
ROA & Retorno sobre los activos & Beneficio neto dividido por los activos totales \\ 
Size & Tamaño de la empresa & Logaritmo natural del total de activos de la empresa \\ 
Debt & Endeudamiento & Pasivos totales divididos por los activos totales \\ 
Growth & Crecimiento de la empresa & Porcentaje de cambio en los activos totales \\ 
VarGDP & Cambio en el PIB & Porcentaje de cambio en el PIB \\ 
Inflat & Inflación del país & Inflación del país en el año \\ 
Gender & Diversidad de género & Porcentaje de mujeres en el consejo de administración \\ 
OpInc & Ingreso operativo & Logaritmo natural del ingreso operativo \\ 
StockT & Rotación del inventario & Costo de ventas dividido por inventario \\ 
AssetT & Rotación de activos & Ingreso operativo dividido por activos totales \\ 
ARP & Período promedio de cobro & Promedio de días que la empresa tarda en recibir pagos \\ 
APP & Período promedio de pago & Promedio de días que la empresa tarda en pagar a proveedores \\ 
Age & Edad de la empresa & Edad de la empresa en años \\ 
LForm & Forma jurídica & La forma jurídica de la empresa: Sociedad anónima, 
    Sociedad limitada, Cooperativa, Otras formas legales \\ 
Country & País de residencia de la empresa & Variable igual a 1 para España y 0 para Italia \\ 
\bottomrule
\end{tabular*}
\end{table}

\textbf{Creacion de varables faltantes}

En esta sección, se calculan nuevas variables derivadas a partir de los
datos existentes en el conjunto DATA\_Manipulada. Estas transformaciones
tienen como objetivo enriquecer el análisis mediante indicadores
adicionales que permiten evaluar distintos aspectos financieros y
operativos de las entidades analizadas. Entre los cálculos realizados,
se incluyen:

\begin{verbatim}
  ·Debt (Endeudamiento):Total de pasivos dividido por el total de activos 
      - Debt = Total_Liabilities / Total_Assets
  ·Oplnc    : Logaritmo natural de los ingresos de explotación 
      - OpInc= log(Operating_Revenue)
  ·StockT(Rotación de inventarios): Ingresos de explotación dividido 
  por el inventario
      - AssetT = Operating_Revenue / Total_Assets 
  ·AssetT   (Rotación de activos): Ingresos de explotación dividido
  por el inventario
      - StockT = Operating_Revenue / Inventory
  ·Age(Antigüedad) :Edad de la empresa en años
      - Age = Seniority / 365 
\end{verbatim}

\begin{Shaded}
\begin{Highlighting}[]
\NormalTok{DATA\_Manipulada}\OtherTok{\textless{}{-}}\NormalTok{ DATA\_Manipulada }\SpecialCharTok{\%\textgreater{}\%}
  \FunctionTok{mutate}\NormalTok{(}\AttributeTok{Debt =}\NormalTok{ Total\_Liabilities }\SpecialCharTok{/}\NormalTok{ Total\_Assets,}
         \AttributeTok{OpInc=} \FunctionTok{log}\NormalTok{(Operating\_Revenue),}
         \AttributeTok{AssetT =}\NormalTok{ Operating\_Revenue }\SpecialCharTok{/}\NormalTok{ Total\_Assets ,}
         \AttributeTok{StockT =}\NormalTok{ Operating\_Revenue }\SpecialCharTok{/}\NormalTok{ Inventory,}
         \AttributeTok{Age =}\NormalTok{ Seniority }\SpecialCharTok{/} \DecValTok{365}\NormalTok{ )}


\CommentTok{\#selección de variables(existentes en la DATA) a utilizar para la Tabla 1}
\NormalTok{DATAM\_SELECT }\OtherTok{\textless{}{-}}\NormalTok{ DATA\_Manipulada }\SpecialCharTok{\%\textgreater{}\%}
  \FunctionTok{select}\NormalTok{(ROE, ROA, Ln\_Total\_Assets,Debt, Growth, GDP\_Var, Inflation, Gender, }
\NormalTok{         OpInc,StockT, AssetT, CollectionPeriod, }
\NormalTok{         Payment\_Period, Age, Legal\_Form, Country) }\SpecialCharTok{\%\textgreater{}\%}
  \FunctionTok{rename}\NormalTok{(}
    \AttributeTok{Size =}\NormalTok{ Ln\_Total\_Assets,}
    \AttributeTok{VarGDP =}\NormalTok{ GDP\_Var,}
    \AttributeTok{Inflat =}\NormalTok{ Inflation,}
    \AttributeTok{ARP =}\NormalTok{ CollectionPeriod,}
    \AttributeTok{APP =}\NormalTok{ Payment\_Period,}
    \AttributeTok{LForm =}\NormalTok{ Legal\_Form) }


\CommentTok{\#aplica a todas las columnas numéricas, redondeando los valores a dos decimales.}
\NormalTok{DATA\_SELECT }\OtherTok{\textless{}{-}}\NormalTok{ DATAM\_SELECT }\SpecialCharTok{\%\textgreater{}\%}
  \FunctionTok{mutate}\NormalTok{(}\FunctionTok{across}\NormalTok{(}\FunctionTok{where}\NormalTok{(is.numeric), }\SpecialCharTok{\textasciitilde{}} \FunctionTok{round}\NormalTok{(.x, }\DecValTok{2}\NormalTok{))) }
\end{Highlighting}
\end{Shaded}

\textbf{SubDataset}

Para facilitar el análisis comparativo, se procedió a dividir el
conjunto de datos principal (DATA\_SELECT) en tres subconjuntos, basados
en la variable Country, que identifica la procedencia de las
observaciones:

\begin{itemize}
\tightlist
\item
  GENERAL: Incluye todas las observaciones del conjunto original,
  excluyendo las variables Country y LForm, ya que no son necesarias
  para este análisis en particular. Este subconjunto representa una
  visión global del dataset.
\item
  ESPAÑA: Contiene únicamente las observaciones donde el valor de
  Country es ``1'', correspondiente a España. Además, se eliminan las
  columnas Country y LForm, manteniendo únicamente las variables
  relevantes para el análisis específico de este país.
\item
  ITALIA: Comprende las observaciones donde el valor de Country es
  ``0'', correspondiente a Italia, siguiendo el mismo criterio de
  selección y limpieza aplicado en el subconjunto de España.
\end{itemize}

\begin{Shaded}
\begin{Highlighting}[]
\CommentTok{\#FILTAR OBSERVACION SEGUN EL PAIS}
\NormalTok{OB\_ES}\OtherTok{\textless{}{-}}\NormalTok{DATA\_SELECT }\SpecialCharTok{\%\textgreater{}\%} \FunctionTok{filter}\NormalTok{(Country}\SpecialCharTok{==}\StringTok{"1"}\NormalTok{) }\CommentTok{\#ESPAÑA}
\NormalTok{OB\_IT}\OtherTok{\textless{}{-}}\NormalTok{DATA\_SELECT }\SpecialCharTok{\%\textgreater{}\%} \FunctionTok{filter}\NormalTok{(Country}\SpecialCharTok{==}\StringTok{"0"}\NormalTok{) }\CommentTok{\#ITALIA}

\NormalTok{GENERAL }\OtherTok{\textless{}{-}}\NormalTok{ DATA\_SELECT }\SpecialCharTok{\%\textgreater{}\%} \FunctionTok{select}\NormalTok{ (}\SpecialCharTok{{-}}\FunctionTok{c}\NormalTok{ (Country, LForm))}
\NormalTok{ESPAÑA }\OtherTok{\textless{}{-}}\NormalTok{OB\_ES }\SpecialCharTok{\%\textgreater{}\%} \FunctionTok{select}\NormalTok{ (}\SpecialCharTok{{-}}\FunctionTok{c}\NormalTok{ (Country, LForm)) }
\NormalTok{ITALIA}\OtherTok{\textless{}{-}}\NormalTok{ OB\_IT }\SpecialCharTok{\%\textgreater{}\%} \FunctionTok{select}\NormalTok{ (}\SpecialCharTok{{-}}\FunctionTok{c}\NormalTok{ (Country, LForm))}
\end{Highlighting}
\end{Shaded}

\section{Estadísticas descriptivas y
correlaciones}\label{estaduxedsticas-descriptivas-y-correlaciones}

\subsection{Tabla 2}\label{tabla-2}

La Tabla 2 presenta las principales estadísticas descriptivas de las
variables utilizadas para explicar la rentabilidad de las empresas que
operan en el sector de la piedra natural en España e Italia durante el
período 2015-2019. Los datos se presentan tanto para la muestra total
como para cada submuestra según el país donde se ubica la empresa.
Además, se incluye la prueba de medias, la cual muestra diferencias
significativas entre ambos países.

Para empezar se reemplazan los valores infinitos por NA para evitar
sesgos en los cálculos. Posteriormente, se generan resúmenes
estadísticos que incluyen la media, desviación estándar, valores mínimos
y máximos de cada variable. Esto permite obtener una visión clara de las
principales características de los datos en cada conjunto. Finalmente,
las tablas son unidas en diferentes combinaciones para su análisis
comparativo.

\begin{Shaded}
\begin{Highlighting}[]
\NormalTok{G }\OtherTok{\textless{}{-}}\NormalTok{ GENERAL }\SpecialCharTok{\%\textgreater{}\%}
  \FunctionTok{mutate}\NormalTok{(}\FunctionTok{across}\NormalTok{(}\FunctionTok{everything}\NormalTok{(), }\SpecialCharTok{\textasciitilde{}} \FunctionTok{ifelse}\NormalTok{(}\FunctionTok{is.infinite}\NormalTok{(.), }\ConstantTok{NA}\NormalTok{, .)))}
\NormalTok{RE\_G}\OtherTok{\textless{}{-}}\NormalTok{ G }\SpecialCharTok{\%\textgreater{}\%} \FunctionTok{get\_summary\_stats}\NormalTok{() }\SpecialCharTok{\%\textgreater{}\%} \FunctionTok{select}\NormalTok{(variable, mean, sd, min, max)}


\NormalTok{E }\OtherTok{\textless{}{-}}\NormalTok{ ESPAÑA }\SpecialCharTok{\%\textgreater{}\%}
  \FunctionTok{mutate}\NormalTok{(}\FunctionTok{across}\NormalTok{(}\FunctionTok{everything}\NormalTok{(), }\SpecialCharTok{\textasciitilde{}} \FunctionTok{ifelse}\NormalTok{(}\FunctionTok{is.infinite}\NormalTok{(.), }\ConstantTok{NA}\NormalTok{, .)))}
\NormalTok{RE\_E }\OtherTok{\textless{}{-}}\NormalTok{ E }\SpecialCharTok{\%\textgreater{}\%} \FunctionTok{get\_summary\_stats}\NormalTok{() }\SpecialCharTok{\%\textgreater{}\%} \FunctionTok{select}\NormalTok{(variable, mean, sd, min, max)}

\NormalTok{I }\OtherTok{\textless{}{-}}\NormalTok{ ITALIA }\SpecialCharTok{\%\textgreater{}\%}
  \FunctionTok{mutate}\NormalTok{(}\FunctionTok{across}\NormalTok{(}\FunctionTok{everything}\NormalTok{(), }\SpecialCharTok{\textasciitilde{}} \FunctionTok{ifelse}\NormalTok{(}\FunctionTok{is.infinite}\NormalTok{(.), }\ConstantTok{NA}\NormalTok{, .)))}
\NormalTok{RE\_I}\OtherTok{\textless{}{-}}\NormalTok{ I }\SpecialCharTok{\%\textgreater{}\%} \FunctionTok{get\_summary\_stats}\NormalTok{() }\SpecialCharTok{\%\textgreater{}\%} \FunctionTok{select}\NormalTok{(variable, mean, sd, min, max)}

\NormalTok{RE\_UNIDO\_E\_I }\OtherTok{\textless{}{-}} \FunctionTok{full\_join}\NormalTok{(RE\_E, RE\_I, }\AttributeTok{by =} \StringTok{"variable"}\NormalTok{)}

\NormalTok{RE\_UNIDO }\OtherTok{\textless{}{-}}\NormalTok{ RE\_G }\SpecialCharTok{\%\textgreater{}\%}
  \FunctionTok{full\_join}\NormalTok{(RE\_E, }\AttributeTok{by =} \StringTok{"variable"}\NormalTok{) }\SpecialCharTok{\%\textgreater{}\%}
  \FunctionTok{full\_join}\NormalTok{(RE\_I, }\AttributeTok{by =} \StringTok{"variable"}\NormalTok{) }

\NormalTok{RE\_UNIDO }\OtherTok{\textless{}{-}}\NormalTok{ RE\_G }\SpecialCharTok{\%\textgreater{}\%}
  \FunctionTok{left\_join}\NormalTok{(RE\_E, }\AttributeTok{by =} \StringTok{"variable"}\NormalTok{) }\SpecialCharTok{\%\textgreater{}\%}
  \FunctionTok{left\_join}\NormalTok{(RE\_I, }\AttributeTok{by =} \StringTok{"variable"}\NormalTok{) }
\end{Highlighting}
\end{Shaded}

se calcula el valor p (p-value) para cada variable mediante una prueba t
de Student, comparando las muestras de España e Italia. Esto permite
evaluar si existen diferencias estadísticamente significativas entre las
dos poblaciones para cada variable. Además, se agrega una columna de
significancia con asteriscos que identifican el nivel de significancia
(p \textless{} 0.10, p \textless{} 0.05, p \textless{} 0.01). Esto
facilita la interpretación de los resultados en términos de relevancia
estadística.

\begin{Shaded}
\begin{Highlighting}[]
\CommentTok{\# Calcular p{-}value para cada variable}
\NormalTok{p\_values }\OtherTok{\textless{}{-}} \FunctionTok{map\_dfr}\NormalTok{(RE\_E}\SpecialCharTok{$}\NormalTok{variable, }\SpecialCharTok{\textasciitilde{}}\NormalTok{ \{}
\NormalTok{  var }\OtherTok{\textless{}{-}}\NormalTok{ .x}
\NormalTok{  t\_test }\OtherTok{\textless{}{-}} \FunctionTok{t.test}\NormalTok{(E[[var]], I[[var]], }\AttributeTok{var.equal =} \ConstantTok{TRUE}\NormalTok{)}
  \FunctionTok{tibble}\NormalTok{(}\AttributeTok{variable =}\NormalTok{ var, }\AttributeTok{p\_value =} \FunctionTok{round}\NormalTok{(t\_test}\SpecialCharTok{$}\NormalTok{p.value, }\DecValTok{2}\NormalTok{))  }
\NormalTok{\})}
\CommentTok{\# Agregar una columna de significancia con asteriscos}
\NormalTok{p\_values }\OtherTok{\textless{}{-}} \FunctionTok{map\_dfr}\NormalTok{(RE\_E}\SpecialCharTok{$}\NormalTok{variable, }\SpecialCharTok{\textasciitilde{}}\NormalTok{ \{}
\NormalTok{  var }\OtherTok{\textless{}{-}}\NormalTok{ .x}
\NormalTok{  t\_test }\OtherTok{\textless{}{-}} \FunctionTok{t.test}\NormalTok{(E[[var]], I[[var]], }\AttributeTok{var.equal =} \ConstantTok{TRUE}\NormalTok{)}
\NormalTok{  p\_value }\OtherTok{\textless{}{-}} \FunctionTok{round}\NormalTok{(t\_test}\SpecialCharTok{$}\NormalTok{p.value, }\DecValTok{4}\NormalTok{)}
  
  \FunctionTok{tibble}\NormalTok{(}
    \AttributeTok{variable =}\NormalTok{ var,}
    \AttributeTok{p\_value =} \FunctionTok{paste0}\NormalTok{(}
      \FunctionTok{format}\NormalTok{(p\_value, }\AttributeTok{nsmall =} \DecValTok{4}\NormalTok{, }\AttributeTok{scientific =} \ConstantTok{FALSE}\NormalTok{),  }
      \FunctionTok{case\_when}\NormalTok{(}
\NormalTok{        p\_value }\SpecialCharTok{\textless{}} \FloatTok{0.01} \SpecialCharTok{\textasciitilde{}} \StringTok{"***"}\NormalTok{,}
\NormalTok{        p\_value }\SpecialCharTok{\textless{}} \FloatTok{0.05} \SpecialCharTok{\textasciitilde{}} \StringTok{"**"}\NormalTok{,}
\NormalTok{        p\_value }\SpecialCharTok{\textless{}} \FloatTok{0.10} \SpecialCharTok{\textasciitilde{}} \StringTok{"*"}\NormalTok{,}
        \ConstantTok{TRUE} \SpecialCharTok{\textasciitilde{}} \StringTok{""}
\NormalTok{      )}
\NormalTok{    )}
\NormalTok{  )}
\NormalTok{\})}
\end{Highlighting}
\end{Shaded}

se combinan los resúmenes estadísticos de España e Italia en una sola
tabla, añadiendo los valores p obtenidos previamente. Luego, esta tabla
es fusionada con las estadísticas descriptivas del conjunto general. El
objetivo es consolidar toda la información en una única tabla que
permita comparar fácilmente las estadísticas generales, españolas e
italianas, junto con los resultados de las pruebas de diferencias de
medias.

\begin{Shaded}
\begin{Highlighting}[]
\CommentTok{\# Combinar las estadísticas descriptivas de España e Italia}
\NormalTok{spain\_italy\_table }\OtherTok{\textless{}{-}}\NormalTok{ RE\_E }\SpecialCharTok{\%\textgreater{}\%}
  \FunctionTok{left\_join}\NormalTok{(RE\_I, }\AttributeTok{by =} \StringTok{"variable"}\NormalTok{, }\AttributeTok{suffix =} \FunctionTok{c}\NormalTok{(}\StringTok{"\_spain"}\NormalTok{, }\StringTok{"\_italy"}\NormalTok{))}

\CommentTok{\# Agregar los p{-}values a la tabla combinada de España e Italia}
\NormalTok{spain\_italy\_table }\OtherTok{\textless{}{-}}\NormalTok{ spain\_italy\_table }\SpecialCharTok{\%\textgreater{}\%}
  \FunctionTok{left\_join}\NormalTok{(p\_values, }\AttributeTok{by =} \StringTok{"variable"}\NormalTok{)}

\CommentTok{\# Combinar la tabla general (RE\_G) con la tabla de España e Italia}
\NormalTok{tabla\_2 }\OtherTok{\textless{}{-}}\NormalTok{ RE\_G }\SpecialCharTok{\%\textgreater{}\%}
  \FunctionTok{left\_join}\NormalTok{(spain\_italy\_table, }\AttributeTok{by =} \StringTok{"variable"}\NormalTok{, }\AttributeTok{suffix =} \FunctionTok{c}\NormalTok{(}\StringTok{"\_total"}\NormalTok{, }\StringTok{""}\NormalTok{))}
\end{Highlighting}
\end{Shaded}

Finalmente, en esta sección, se formatea la tabla combinada utilizando
la librería gt para generar una presentación clara y profesional. Se
ajustan los números a cuatro decimales, se renombran las columnas para
mayor claridad y se agrupan las estadísticas bajo encabezados
representativos (``Total Sample'', ``Spain'', ``Italy''). Asimismo, se
agrega un encabezado para la tabla, resaltando que se trata de
estadísticas descriptivas y pruebas de diferencias de medias por país.
Este formato facilita la interpretación visual de los resultados.

\begin{Shaded}
\begin{Highlighting}[]
\CommentTok{\# Formatear la tabla con gt()}
\NormalTok{tabla\_2 }\SpecialCharTok{\%\textgreater{}\%}
  \FunctionTok{gt}\NormalTok{() }\SpecialCharTok{\%\textgreater{}\%}
  \FunctionTok{fmt\_number}\NormalTok{(}
    \AttributeTok{columns =} \FunctionTok{c}\NormalTok{(mean, sd, min, max,}
\NormalTok{                mean\_spain, sd\_spain, min\_spain, max\_spain,}
\NormalTok{                mean\_italy, sd\_italy, min\_italy, max\_italy),}
    \AttributeTok{decimals =} \DecValTok{2}
\NormalTok{  ) }\SpecialCharTok{\%\textgreater{}\%}
\FunctionTok{cols\_label}\NormalTok{(}
  \AttributeTok{variable =} \FunctionTok{md}\NormalTok{(}\StringTok{"**Variable**"}\NormalTok{),}
  \AttributeTok{mean =} \FunctionTok{md}\NormalTok{(}\StringTok{"**M**"}\NormalTok{),}
  \AttributeTok{sd =} \FunctionTok{md}\NormalTok{(}\StringTok{"**SD**"}\NormalTok{),}
  \AttributeTok{min =} \FunctionTok{md}\NormalTok{(}\StringTok{"**Min**"}\NormalTok{),}
  \AttributeTok{max =} \FunctionTok{md}\NormalTok{(}\StringTok{"**Max**"}\NormalTok{),}
  \AttributeTok{mean\_spain =} \FunctionTok{md}\NormalTok{(}\StringTok{"**M**"}\NormalTok{),}
  \AttributeTok{sd\_spain =} \FunctionTok{md}\NormalTok{(}\StringTok{"**SD**"}\NormalTok{),}
  \AttributeTok{min\_spain =} \FunctionTok{md}\NormalTok{(}\StringTok{"**Min**"}\NormalTok{),}
  \AttributeTok{max\_spain =} \FunctionTok{md}\NormalTok{(}\StringTok{"**Max**"}\NormalTok{),}
  \AttributeTok{mean\_italy =} \FunctionTok{md}\NormalTok{(}\StringTok{"**M**"}\NormalTok{),}
  \AttributeTok{sd\_italy =} \FunctionTok{md}\NormalTok{(}\StringTok{"**SD**"}\NormalTok{),}
  \AttributeTok{min\_italy =} \FunctionTok{md}\NormalTok{(}\StringTok{"**Min**"}\NormalTok{),}
  \AttributeTok{max\_italy =} \FunctionTok{md}\NormalTok{(}\StringTok{"**Max**"}\NormalTok{),}
  \AttributeTok{p\_value =} \FunctionTok{md}\NormalTok{(}\StringTok{"**valor p**"}\NormalTok{)}
\NormalTok{) }\SpecialCharTok{\%\textgreater{}\%}
  \FunctionTok{tab\_spanner}\NormalTok{(}
    \AttributeTok{label =} \FunctionTok{md}\NormalTok{(}\StringTok{"**Muestra Total**"}\NormalTok{),}
    \AttributeTok{columns =} \FunctionTok{c}\NormalTok{(mean, sd, min, max)}
\NormalTok{  ) }\SpecialCharTok{\%\textgreater{}\%}
  \FunctionTok{tab\_spanner}\NormalTok{(}
    \AttributeTok{label =} \FunctionTok{md}\NormalTok{(}\StringTok{"**España**"}\NormalTok{),}
    \AttributeTok{columns =} \FunctionTok{c}\NormalTok{(mean\_spain, sd\_spain, min\_spain, max\_spain)}
\NormalTok{  ) }\SpecialCharTok{\%\textgreater{}\%}
  \FunctionTok{tab\_spanner}\NormalTok{(}
    \AttributeTok{label =} \FunctionTok{md}\NormalTok{(}\StringTok{"**Italia**"}\NormalTok{),}
    \AttributeTok{columns =} \FunctionTok{c}\NormalTok{(mean\_italy, sd\_italy, min\_italy, max\_italy)}
\NormalTok{  ) }\SpecialCharTok{\%\textgreater{}\%}
  \FunctionTok{tab\_header}\NormalTok{(}
    \AttributeTok{title =} \FunctionTok{md}\NormalTok{(}\StringTok{"**Tabla 2. Estadísticas descriptivas y prueba de }
\StringTok{               diferencia de medias por país**"}\NormalTok{)}
\NormalTok{  )}\SpecialCharTok{\%\textgreater{}\%}
  \FunctionTok{tab\_options}\NormalTok{(}
    \AttributeTok{column\_labels.font.size =} \FunctionTok{px}\NormalTok{(}\DecValTok{7}\NormalTok{), }\CommentTok{\# Tamaño del texto de encabezado}
    \AttributeTok{table.font.size =} \FunctionTok{px}\NormalTok{(}\DecValTok{7}\NormalTok{) }\CommentTok{\# Tamaño del texto en la tabla}
\NormalTok{  )}\SpecialCharTok{\%\textgreater{}\%}
  \FunctionTok{tab\_style}\NormalTok{(}
    \AttributeTok{style =} \FunctionTok{list}\NormalTok{(}
      \FunctionTok{cell\_text}\NormalTok{(}\AttributeTok{align =} \StringTok{"center"}\NormalTok{)  }
\NormalTok{    ),}
    \AttributeTok{locations =} \FunctionTok{cells\_title}\NormalTok{(}\AttributeTok{groups =} \StringTok{"title"}\NormalTok{) }
\NormalTok{  )}
\end{Highlighting}
\end{Shaded}

\begin{table}[!t]
\caption*{
{\large \textbf{Tabla 2. Estadísticas descriptivas y prueba de
diferencia de medias por país}}
} 
\fontsize{5.2pt}{6.3pt}\selectfont
\begin{tabular*}{\linewidth}{@{\extracolsep{\fill}}crrrrrrrrrrrrr}
\toprule
 & \multicolumn{4}{c}{\textbf{Muestra Total}} & \multicolumn{4}{c}{\textbf{España}} & \multicolumn{4}{c}{\textbf{Italia}} &  \\ 
\cmidrule(lr){2-5} \cmidrule(lr){6-9} \cmidrule(lr){10-13}
\textbf{Variable} & \textbf{M} & \textbf{SD} & \textbf{Min} & \textbf{Max} & \textbf{M} & \textbf{SD} & \textbf{Min} & \textbf{Max} & \textbf{M} & \textbf{SD} & \textbf{Min} & \textbf{Max} & \textbf{valor p} \\ 
\midrule\addlinespace[2.5pt]
ROE & 4.86 & 50.63 & -973.58 & 169.28 & 2.81 & 55.56 & -920.19 & 164.48 & 6.47 & 46.38 & -973.58 & 169.28 & 0.0992* \\ 
ROA & 2.99 & 8.33 & -79.25 & 65.44 & 2.96 & 8.79 & -79.25 & 65.44 & 3.02 & 7.95 & -65.89 & 46.23 & 0.8717 \\ 
Size & 8.25 & 1.10 & 1.10 & 12.63 & 8.25 & 1.26 & 1.10 & 12.63 & 8.25 & 0.95 & 2.75 & 12.26 & 0.9788 \\ 
Debt & 0.54 & 0.31 & 0.00 & 2.59 & 0.44 & 0.30 & 0.00 & 1.86 & 0.61 & 0.30 & 0.00 & 2.59 & 0.0000*** \\ 
Growth & 0.64 & 15.68 & -1.00 & 479.78 & 0.66 & 17.00 & -1.00 & 479.78 & 0.62 & 14.55 & -1.00 & 454.58 & 0.9599 \\ 
VarGDP & 1.82 & 1.07 & 0.02 & 3.84 & 2.83 & 0.65 & 0.02 & 3.84 & 0.99 & 0.47 & 0.29 & 1.67 & 0.0000*** \\ 
Inflat & 0.65 & 0.77 & -0.50 & 1.96 & 0.73 & 0.98 & -0.50 & 1.96 & 0.59 & 0.54 & -0.09 & 1.23 & 0.0000*** \\ 
Gender & 21.38 & 31.00 & 0.00 & 100.00 & 24.36 & 30.47 & 0.00 & 100.00 & 18.95 & 31.22 & 0.00 & 100.00 & 0.0000*** \\ 
OpInc & 7.48 & 1.30 & -1.82 & 12.43 & 7.55 & 1.26 & 1.27 & 12.43 & 7.42 & 1.33 & -1.82 & 11.92 & 0.0171** \\ 
StockT & 66.77 & 353.73 & -0.15 & 9,680.32 & 47.80 & 254.31 & 0.03 & 3,647.51 & 82.09 & 416.62 & -0.15 & 9,680.32 & 0.0387** \\ 
AssetT & 0.69 & 0.70 & -0.06 & 15.20 & 0.77 & 0.91 & 0.00 & 15.20 & 0.63 & 0.46 & -0.06 & 3.69 & 0.0000*** \\ 
ARP & 132.06 & 140.49 & 0.00 & 981.75 & 132.71 & 130.10 & 0.00 & 967.13 & 131.52 & 148.53 & 0.00 & 981.75 & 0.8432 \\ 
APP & 64.92 & 104.16 & 0.00 & 993.43 & 47.50 & 77.82 & 0.00 & 993.43 & 79.21 & 119.74 & 0.00 & 962.24 & 0.0000*** \\ 
Age & 27.84 & 17.11 & -2.95 & 103.41 & 26.45 & 12.73 & -2.95 & 62.05 & 28.96 & 19.91 & -2.55 & 103.41 & 0.0005*** \\ 
\bottomrule
\end{tabular*}
\end{table}

\subsection{Tabla 3}\label{tabla-3}

La Tabla 3 muestra los resultados de la prueba de chi-cuadrado aplicada
a la forma legal según el país de referencia. Se encuentran diferencias
significativas entre España e Italia. Mientras que en España hay muchas
empresas con la forma legal de sociedad anónima, en Italia se observa
una clara preferencia por las sociedades de responsabilidad limitada.
Del mismo modo, Italia cuenta con cooperativas y otras formas sociales
en este sector, las cuales prácticamente no existen en España. De hecho,
España es un país donde la forma legal de cooperativa es ampliamente
utilizada en otros sectores, como la agricultura, pero está casi ausente
en el sector de la piedra natural.

Se tiene como propósito contar las frecuencias de los niveles de la
variable categórica LForm (que representa la forma legal de las
empresas) en diferentes conjuntos de datos.

En primer lugar, se trabaja con el conjunto de datos DATA\_SELECT, que
contiene información general. A través de la función count(LForm), se
calcula el número total de empresas para cada nivel de la variable
LForm. Posteriormente, se renombra la columna que contiene los conteos
como ``Total sample'' para indicar que representa la totalidad de la
muestra.

\begin{Shaded}
\begin{Highlighting}[]
\NormalTok{LF\_GENERAL }\OtherTok{\textless{}{-}}\NormalTok{ DATA\_SELECT }\SpecialCharTok{\%\textgreater{}\%} \FunctionTok{count}\NormalTok{(LForm) }\SpecialCharTok{\%\textgreater{}\%} \FunctionTok{rename}\NormalTok{(}\StringTok{"Total sample"} \OtherTok{=}\NormalTok{ n)}
\NormalTok{LF\_ESPAÑA}\OtherTok{\textless{}{-}}\NormalTok{ OB\_ES }\SpecialCharTok{\%\textgreater{}\%} \FunctionTok{count}\NormalTok{(LForm) }\SpecialCharTok{\%\textgreater{}\%} \FunctionTok{rename}\NormalTok{(}\AttributeTok{Spain =}\NormalTok{ n)}
\NormalTok{LF\_ITALIA}\OtherTok{\textless{}{-}}\NormalTok{ OB\_IT }\SpecialCharTok{\%\textgreater{}\%} \FunctionTok{count}\NormalTok{(LForm) }\SpecialCharTok{\%\textgreater{}\%} \FunctionTok{rename}\NormalTok{(}\AttributeTok{Italy =}\NormalTok{ n)}
\end{Highlighting}
\end{Shaded}

Se utiliza la función full\_join para unir tres tablas basadas en una
columna común. La unión completa asegura que todas las observaciones de
las tablas se conserven, incluso si no hay coincidencias entre ellas.
Luego, se usa is.na() para identificar y reemplazar los valores
faltantes con ceros en el resultado de la unión.

\begin{Shaded}
\begin{Highlighting}[]
\CommentTok{\# Unir las tablas por \textquotesingle{}LForm\textquotesingle{} (el tipo de empresa)}
\NormalTok{T\_LEGAL\_FORM }\OtherTok{\textless{}{-}}\NormalTok{ LF\_GENERAL }\SpecialCharTok{\%\textgreater{}\%}
  \FunctionTok{full\_join}\NormalTok{(LF\_ESPAÑA, }\AttributeTok{by =} \StringTok{"LForm"}\NormalTok{) }\SpecialCharTok{\%\textgreater{}\%}
  \FunctionTok{full\_join}\NormalTok{(LF\_ITALIA, }\AttributeTok{by =} \StringTok{"LForm"}\NormalTok{)}
\end{Highlighting}
\end{Shaded}

Posteriormente, se emplean las funciones distinct para obtener los
valores únicos de dos columnas relacionadas con las formas legales. Se
define un vector que asigna nombres descriptivos a códigos numéricos
representando tipos de empresas. Finalmente, la función mutate se usa
para reemplazar los códigos numéricos por los nombres de los tipos de
empresas, y rename cambia el nombre de la columna LForm a ``Legal form''
para una mejor comprensión.

\begin{Shaded}
\begin{Highlighting}[]
\CommentTok{\# Rellenar con ceros en caso de valores faltantes}
\NormalTok{T\_LEGAL\_FORM[}\FunctionTok{is.na}\NormalTok{(T\_LEGAL\_FORM)] }\OtherTok{\textless{}{-}} \DecValTok{0}

\NormalTok{DATA\_Manipulada }\SpecialCharTok{\%\textgreater{}\%} \FunctionTok{distinct}\NormalTok{(Standard\_Legal\_Form)}
\NormalTok{DATA\_Manipulada }\SpecialCharTok{\%\textgreater{}\%} \FunctionTok{distinct}\NormalTok{(Legal\_Form)}
\NormalTok{tipo\_empresa }\OtherTok{\textless{}{-}} \FunctionTok{c}\NormalTok{(}\StringTok{"0"} \OtherTok{=} \StringTok{"Public limited company"}\NormalTok{, }
                  \StringTok{"1"} \OtherTok{=} \StringTok{"Private limited company"}\NormalTok{, }
                  \StringTok{"3"} \OtherTok{=} \StringTok{"Cooperative"}\NormalTok{, }
                  \StringTok{"4"} \OtherTok{=} \StringTok{"Other legal forms"}\NormalTok{)}
\CommentTok{\# Reemplazar los códigos numéricos con los nombres de tipos de empresas}
\NormalTok{T\_LEGAL\_FORM }\OtherTok{\textless{}{-}}\NormalTok{T\_LEGAL\_FORM }\SpecialCharTok{\%\textgreater{}\%}
  \FunctionTok{mutate}\NormalTok{(}\AttributeTok{LForm =}\NormalTok{ tipo\_empresa[}\FunctionTok{as.character}\NormalTok{(LForm)])}\SpecialCharTok{\%\textgreater{}\%} 
  \FunctionTok{rename}\NormalTok{(}\StringTok{"Legal form"} \OtherTok{=}\NormalTok{ LForm)}
\end{Highlighting}
\end{Shaded}

Se realiza el calculo del test de Chi-cuadrado para comparar las
frecuencias observadas entre dos columnas específicas (en este caso,
España e Italia). Primero, selecciona los valores de estas dos columnas
y los convierte en una matriz mediante as.matrix(). Luego, se aplica la
función chisq.test() para calcular el test de Chi-cuadrado sobre las
frecuencias observadas.

\begin{Shaded}
\begin{Highlighting}[]
\CommentTok{\#Calculo de Chi{-}squared test}
\NormalTok{observed }\OtherTok{\textless{}{-}}\NormalTok{ T\_LEGAL\_FORM }\SpecialCharTok{\%\textgreater{}\%}
  \FunctionTok{select}\NormalTok{(Spain, Italy) }\SpecialCharTok{\%\textgreater{}\%}
  \FunctionTok{as.matrix}\NormalTok{()}

\NormalTok{chisq\_test }\OtherTok{\textless{}{-}} \FunctionTok{chisq.test}\NormalTok{(observed)}

\CommentTok{\# Extraer los valores del test y formatearlos}
\NormalTok{chi\_squared\_value }\OtherTok{\textless{}{-}} \FunctionTok{round}\NormalTok{(chisq\_test}\SpecialCharTok{$}\NormalTok{statistic, }\DecValTok{4}\NormalTok{)}

\CommentTok{\# Formatear p{-}value manualmente (sin "\textless{}")}
\NormalTok{p\_value }\OtherTok{\textless{}{-}} \FunctionTok{ifelse}\NormalTok{(chisq\_test}\SpecialCharTok{$}\NormalTok{p.value }\SpecialCharTok{\textless{}} \FloatTok{0.0000}\NormalTok{, }\StringTok{"0.0000"}\NormalTok{, }
                  \FunctionTok{sprintf}\NormalTok{(}\StringTok{"\%.4f"}\NormalTok{, chisq\_test}\SpecialCharTok{$}\NormalTok{p.value)) }

\CommentTok{\# Agregar el resultado del Chi{-}cuadrado a la tabla}
\NormalTok{tabla\_3 }\OtherTok{\textless{}{-}}\NormalTok{ T\_LEGAL\_FORM }\SpecialCharTok{\%\textgreater{}\%}
  \FunctionTok{mutate}\NormalTok{(}\StringTok{\textasciigrave{}}\AttributeTok{Chi{-}squared test}\StringTok{\textasciigrave{}} \OtherTok{=} \FunctionTok{ifelse}\NormalTok{(}\FunctionTok{row\_number}\NormalTok{() }\SpecialCharTok{==} \DecValTok{1}\NormalTok{, }
                                     \FunctionTok{paste0}\NormalTok{(chi\_squared\_value, }
                                            \StringTok{" ("}\NormalTok{, p\_value, }\StringTok{")"}\NormalTok{), }
                                     \StringTok{""}\NormalTok{))}
\end{Highlighting}
\end{Shaded}

Finalmente se utiliza la función gt() para crear una tabla estilizada,
personalizándola con opciones para ajustar títulos, etiquetas de
columnas, alineación de celdas, bordes y notas al pie.

La Tabla 3 muestre los resultados del test de Chi-cuadrado junto con las
frecuencias de las formas legales por país, mejorando así la
presentación de los datos y facilitando su comprensión.

\begin{Shaded}
\begin{Highlighting}[]
\CommentTok{\# Formatear la tabla con gt()}
\NormalTok{tabla\_3 }\SpecialCharTok{\%\textgreater{}\%}
  \FunctionTok{gt}\NormalTok{() }\SpecialCharTok{\%\textgreater{}\%}
  \FunctionTok{tab\_header}\NormalTok{(}
    \AttributeTok{title =} \FunctionTok{md}\NormalTok{(}\StringTok{"**Tabla 3. Forma jurídica por país**"}\NormalTok{),}
    \AttributeTok{subtitle =} \StringTok{""}
\NormalTok{  ) }\SpecialCharTok{\%\textgreater{}\%}
  \FunctionTok{cols\_label}\NormalTok{(}
    \StringTok{\textasciigrave{}}\AttributeTok{Legal form}\StringTok{\textasciigrave{}} \OtherTok{=} \FunctionTok{md}\NormalTok{(}\StringTok{"**Forma jurídica**"}\NormalTok{),}
    \StringTok{\textasciigrave{}}\AttributeTok{Total sample}\StringTok{\textasciigrave{}} \OtherTok{=} \FunctionTok{md}\NormalTok{(}\StringTok{"**Muestra total**"}\NormalTok{),}
    \AttributeTok{Spain =} \FunctionTok{md}\NormalTok{(}\StringTok{"**España**"}\NormalTok{),}
    \AttributeTok{Italy =} \FunctionTok{md}\NormalTok{(}\StringTok{"**Italia**"}\NormalTok{),}
    \StringTok{\textasciigrave{}}\AttributeTok{Chi{-}squared test}\StringTok{\textasciigrave{}} \OtherTok{=} \FunctionTok{md}\NormalTok{(}\StringTok{"**Prueba de chi{-}cuadrado**"}\NormalTok{)}
\NormalTok{  ) }\SpecialCharTok{\%\textgreater{}\%}
  \FunctionTok{cols\_align}\NormalTok{(}
    \AttributeTok{align =} \StringTok{"left"}\NormalTok{,}
    \AttributeTok{columns =} \StringTok{\textasciigrave{}}\AttributeTok{Legal form}\StringTok{\textasciigrave{}}
\NormalTok{  ) }\SpecialCharTok{\%\textgreater{}\%}
  \FunctionTok{cols\_align}\NormalTok{(}
    \AttributeTok{align =} \StringTok{"center"}\NormalTok{,}
    \AttributeTok{columns =} \FunctionTok{c}\NormalTok{(}\StringTok{\textasciigrave{}}\AttributeTok{Total sample}\StringTok{\textasciigrave{}}\NormalTok{, Spain, Italy, }\StringTok{\textasciigrave{}}\AttributeTok{Chi{-}squared test}\StringTok{\textasciigrave{}}\NormalTok{)}
\NormalTok{  )}\SpecialCharTok{\%\textgreater{}\%}
  \FunctionTok{tab\_options}\NormalTok{(}
    \AttributeTok{column\_labels.font.size =} \FunctionTok{px}\NormalTok{(}\DecValTok{12}\NormalTok{), }\CommentTok{\# Tamaño del texto de encabezado}
    \AttributeTok{table.font.size =} \FunctionTok{px}\NormalTok{(}\DecValTok{12}\NormalTok{) }\CommentTok{\# Tamaño del texto en la tabla}
\NormalTok{  )}\SpecialCharTok{\%\textgreater{}\%}
  \FunctionTok{tab\_style}\NormalTok{(}
    \AttributeTok{style =} \FunctionTok{list}\NormalTok{(}
      \FunctionTok{cell\_text}\NormalTok{(}\AttributeTok{align =} \StringTok{"center"}\NormalTok{)  }
\NormalTok{    ),}
    \AttributeTok{locations =} \FunctionTok{cells\_title}\NormalTok{(}\AttributeTok{groups =} \StringTok{"title"}\NormalTok{)  }
\NormalTok{  )}
\end{Highlighting}
\end{Shaded}

\begin{table}[!t]
\caption*{
{\large \textbf{Tabla 3. Forma jurídica por país}}
} 
\fontsize{9.0pt}{10.8pt}\selectfont
\begin{tabular*}{\linewidth}{@{\extracolsep{\fill}}lcccc}
\toprule
\textbf{Forma jurídica} & \textbf{Muestra total} & \textbf{España} & \textbf{Italia} & \textbf{Prueba de chi-cuadrado} \\ 
\midrule\addlinespace[2.5pt]
Public limited company & 405 & 325 & 80 & 272.8377 (0.0000) \\ 
Private limited company & 1765 & 670 & 1095 &  \\ 
Cooperative & 65 & 20 & 45 &  \\ 
Other legal forms & 35 & 0 & 35 &  \\ 
\bottomrule
\end{tabular*}
\end{table}

\subsection{Tabla 4}\label{tabla-4}

Muestra la matriz de correlación de Pearson entre las variables
continuas utilizadas en el estudio empírico. Se puede observar que no
existen correlaciones elevadas entre los regresores que puedan generar
problemas de colinealidad en el análisis multivariante posterior.

Además, todos los regresores, excepto la inflación, presentan una
correlación significativa con las variables dependientes (ROE y ROA).
Específicamente, el volumen de activos, los períodos promedio de cobro y
pago, la antigüedad de la empresa, el cambio en el PIB y el género de la
dirección están negativamente correlacionados con ROE y ROA. Para el
cambio en el PIB y el género, la relación solo es significativa en el
caso del ROE.

Por otro lado, la correlación es positiva para las variables que miden
el ingreso operativo, la rotación de inventario, la rotación de activos
y el crecimiento, aunque solo la correlación entre ROA y crecimiento es
significativa.

En cuanto al endeudamiento, la correlación es significativa y negativa
para el ROA, pero positiva para el ROE.

Para calcular la matriz de correlación, primero se usa select(), que
permite extraer únicamente las variables numéricas de interés. Luego,
con cor\_mat(method = ``pearson''), se calcula la matriz de correlación
utilizando el método de Pearson, que mide la relación lineal entre las
variables.

Posteriormente, cor\_mark\_significant() se encarga de marcar aquellas
correlaciones que son estadísticamente significativas, lo que ayuda a
interpretar los resultados.

\begin{Shaded}
\begin{Highlighting}[]
\CommentTok{\# Generar la matriz de correlación}
\NormalTok{correlation\_matrix }\OtherTok{\textless{}{-}}\NormalTok{ G }\SpecialCharTok{\%\textgreater{}\%}
  \FunctionTok{select}\NormalTok{(ROE, ROA, Size, Debt, Growth, VarGDP, Inflat, Gender, OpInc, StockT, }
\NormalTok{         AssetT, ARP, APP, Age) }\SpecialCharTok{\%\textgreater{}\%}
  \FunctionTok{cor\_mat}\NormalTok{(}\AttributeTok{method =} \StringTok{"pearson"}\NormalTok{) }\SpecialCharTok{\%\textgreater{}\%}
  \FunctionTok{cor\_mark\_significant}\NormalTok{() }\CommentTok{\# Añade los niveles de significancia}
\end{Highlighting}
\end{Shaded}

Finalmente, para visualizar los datos de manera clara, se utiliza gt(),
que convierte la matriz en una tabla interactiva, y tab\_options(), que
ajusta el tamaño del texto de la tabla para mejorar la presentación.

\begin{Shaded}
\begin{Highlighting}[]
\CommentTok{\# Mostrar la matriz de correlación en una vista interactiva}
\NormalTok{correlation\_matrix  }\SpecialCharTok{\%\textgreater{}\%} 
  \FunctionTok{gt}\NormalTok{() }\SpecialCharTok{\%\textgreater{}\%}
  \FunctionTok{tab\_header}\NormalTok{(}
    \AttributeTok{title =} \FunctionTok{md}\NormalTok{(}\StringTok{"**Tabla 4. Correlación pearson para las variables continuas**"}\NormalTok{)}
\NormalTok{  ) }\SpecialCharTok{\%\textgreater{}\%}
  \FunctionTok{tab\_options}\NormalTok{(}
    \AttributeTok{column\_labels.font.size =} \FunctionTok{px}\NormalTok{(}\DecValTok{5}\NormalTok{), }\CommentTok{\# Tamaño del texto de encabezado}
    \AttributeTok{table.font.size =} \FunctionTok{px}\NormalTok{(}\DecValTok{5}\NormalTok{) }\CommentTok{\# Tamaño del texto en la tabla}
\NormalTok{  )}\SpecialCharTok{\%\textgreater{}\%}
  \FunctionTok{tab\_style}\NormalTok{(}
    \AttributeTok{style =} \FunctionTok{list}\NormalTok{(}
      \FunctionTok{cell\_text}\NormalTok{(}\AttributeTok{align =} \StringTok{"center"}\NormalTok{)  }
\NormalTok{    ),}
    \AttributeTok{locations =} \FunctionTok{cells\_title}\NormalTok{(}\AttributeTok{groups =} \StringTok{"title"}\NormalTok{)  }
\NormalTok{  )}
\end{Highlighting}
\end{Shaded}

\begin{table}[!t]
\caption*{
{\large \textbf{Tabla 4. Correlación pearson para las variables continuas}}
} 
\fontsize{3.8pt}{4.5pt}\selectfont
\begin{tabular*}{\linewidth}{@{\extracolsep{\fill}}l|rrrrrrrrrrrrrr}
\toprule
 & ROE & ROA & Size & Debt & Growth & VarGDP & Inflat & Gender & OpInc & StockT & AssetT & ARP & APP & Age \\ 
\midrule\addlinespace[2.5pt]
ROE &  &  &  &  &  &  &  &  &  &  &  &  &  &  \\ 
ROA & 0.44**** &  &  &  &  &  &  &  &  &  &  &  &  &  \\ 
Size & -0.064** & -0.064** &  &  &  &  &  &  &  &  &  &  &  &  \\ 
Debt & -0.07** & -0.24**** & -0.2**** &  &  &  &  &  &  &  &  &  &  &  \\ 
Growth & 0.02 & 0.059* & -0.012 & -0.013 &  &  &  &  &  &  &  &  &  &  \\ 
VarGDP & -0.025 & -0.0083 & -0.02 & -0.22**** & 0.0043 &  &  &  &  &  &  &  &  &  \\ 
Inflat & -0.024 & 0.0041 & 0.028 & -0.033 & -0.00053 & -0.055** &  &  &  &  &  &  &  &  \\ 
Gender & 0.0084 & -0.012 & -0.048* & -0.046* & -0.024 & 0.072*** & 0.0076 &  &  &  &  &  &  &  \\ 
OpInc & 0.071** & 0.2**** & 0.58**** & -0.1**** & 0.0028 & 0.013 & 0.038 & -0.078*** &  &  &  &  &  &  \\ 
StockT & 0.068** & 0.12**** & -0.011 & 0.018 & 0.018 & -0.053* & -0.032 & -0.0082 & 0.067** &  &  &  &  &  \\ 
AssetT & 0.13**** & 0.3**** & -0.43**** & 0.15**** & 0.019 & 0.083*** & 0.0062 & -0.029 & 0.21**** & 0.11**** &  &  &  &  \\ 
ARP & -0.07** & -0.17**** & 0.09**** & -0.022 & 0.012 & 0.0094 & -0.011 & -0.031 & -0.16**** & -0.0067 & -0.25**** &  &  &  \\ 
APP & -0.1**** & -0.19**** & -0.034 & 0.2**** & -0.0042 & -0.13**** & -0.031 & -0.051* & -0.2**** & -0.033 & -0.15**** & 0.32**** &  &  \\ 
Age & -0.0053 & -0.061** & 0.32**** & -0.17**** & -0.06* & -0.036 & -0.056** & 0.07*** & 0.14**** & -0.039 & -0.21**** & 0.034 & -0.045* &  \\ 
\bottomrule
\end{tabular*}
\end{table}

\subsection{Tabla 5}\label{tabla-5}

Muestra los resultados del análisis de varianza del ROA y ROE en función
de la forma jurídica de la empresa. Se observa que, efectivamente,
existe una relación significativa entre la forma jurídica de la empresa
y los dos tipos de rentabilidad, tanto en España como en Italia.

Se calcula el promedio de los indicadores financieros ROE (Return on
Equity) y ROA (Return on Assets) según la forma legal de las empresas.
Este análisis se realiza de manera general, considerando toda la muestra
de datos, y luego se repite específicamente para empresas en España e
Italia. Para ello, se utiliza la función group\_by() para agrupar los
datos según la forma legal (LForm), seguida de summarise() para calcular
la media de ROE y ROA, excluyendo valores nulos mediante na.rm = TRUE.
Estos cálculos permiten obtener una visión comparativa del desempeño
financiero según la estructura legal de las empresas en diferentes
regiones.

\begin{Shaded}
\begin{Highlighting}[]
\CommentTok{\# Calcular promedios de ROE y ROA por tipo de empresa (Legal Form) GENRAL}
\NormalTok{legal\_form\_means\_GENERAL }\OtherTok{\textless{}{-}}\NormalTok{ DATA\_SELECT }\SpecialCharTok{\%\textgreater{}\%} \FunctionTok{select}\NormalTok{(LForm , Country, ROA, ROE)}\SpecialCharTok{\%\textgreater{}\%} 
  \FunctionTok{group\_by}\NormalTok{(LForm) }\SpecialCharTok{\%\textgreater{}\%}
  \FunctionTok{summarise}\NormalTok{(}
    \AttributeTok{Mean\_ROE =} \FunctionTok{mean}\NormalTok{(ROE, }\AttributeTok{na.rm =} \ConstantTok{TRUE}\NormalTok{),}
    \AttributeTok{Mean\_ROA =} \FunctionTok{mean}\NormalTok{(ROA, }\AttributeTok{na.rm =} \ConstantTok{TRUE}\NormalTok{)}
\NormalTok{  )}
\CommentTok{\# Calcular promedios de ROE y ROA por tipo de empresa (Legal Form) ESPAÑA}
\NormalTok{legal\_form\_means\_ESPAÑA }\OtherTok{\textless{}{-}}\NormalTok{ OB\_ES }\SpecialCharTok{\%\textgreater{}\%} \FunctionTok{select}\NormalTok{(LForm , Country, ROA, ROE) }\SpecialCharTok{\%\textgreater{}\%} 
  \FunctionTok{group\_by}\NormalTok{(LForm) }\SpecialCharTok{\%\textgreater{}\%}
  \FunctionTok{summarise}\NormalTok{(}
    \AttributeTok{Mean\_ROE =} \FunctionTok{mean}\NormalTok{(ROE, }\AttributeTok{na.rm =} \ConstantTok{TRUE}\NormalTok{),}
    \AttributeTok{Mean\_ROA =} \FunctionTok{mean}\NormalTok{(ROA, }\AttributeTok{na.rm =} \ConstantTok{TRUE}\NormalTok{)}
\NormalTok{  )}
\CommentTok{\# Calcular promedios de ROE y ROA por tipo de empresa (Legal Form) ITALIA}
\NormalTok{legal\_form\_means\_ITALIA }\OtherTok{\textless{}{-}}\NormalTok{ OB\_IT }\SpecialCharTok{\%\textgreater{}\%} \FunctionTok{select}\NormalTok{(LForm , Country, ROA, ROE) }\SpecialCharTok{\%\textgreater{}\%} 
  \FunctionTok{group\_by}\NormalTok{(LForm) }\SpecialCharTok{\%\textgreater{}\%}
  \FunctionTok{summarise}\NormalTok{(}
    \AttributeTok{Mean\_ROE =} \FunctionTok{mean}\NormalTok{(ROE, }\AttributeTok{na.rm =} \ConstantTok{TRUE}\NormalTok{),}
    \AttributeTok{Mean\_ROA =} \FunctionTok{mean}\NormalTok{(ROA, }\AttributeTok{na.rm =} \ConstantTok{TRUE}\NormalTok{)}
\NormalTok{  )}
\end{Highlighting}
\end{Shaded}

Se realiza un análisis de varianza (ANOVA) para evaluar si existen
diferencias estadísticamente significativas en los valores de ROA y ROE
según la forma legal de las empresas. Este análisis se lleva a cabo en
tres niveles: para toda la muestra, para España y para Italia. Se
utiliza la función aov(), que aplica un ANOVA unidireccional sobre la
variable dependiente (ROA o ROE) en función de la variable categórica
LForm. Los resultados de este análisis permitirán determinar si la
estructura legal de una empresa influye significativamente en su
desempeño financiero en cada contexto geográfico.

\begin{Shaded}
\begin{Highlighting}[]
\CommentTok{\# ANOVA por forma legal para ROA y ROE (Muestra total)}
\NormalTok{anova\_roa\_total }\OtherTok{\textless{}{-}} \FunctionTok{aov}\NormalTok{(ROA }\SpecialCharTok{\textasciitilde{}}\NormalTok{ LForm, }\AttributeTok{data =}\NormalTok{ DATA\_SELECT)}
\NormalTok{anova\_roe\_total }\OtherTok{\textless{}{-}} \FunctionTok{aov}\NormalTok{(ROE }\SpecialCharTok{\textasciitilde{}}\NormalTok{ LForm, }\AttributeTok{data =}\NormalTok{ DATA\_SELECT)}
\CommentTok{\# ANOVA por forma legal para España}
\NormalTok{anova\_roa\_es }\OtherTok{\textless{}{-}} \FunctionTok{aov}\NormalTok{(ROA }\SpecialCharTok{\textasciitilde{}}\NormalTok{ LForm, }\AttributeTok{data =}\NormalTok{ OB\_ES)}
\NormalTok{anova\_roe\_es }\OtherTok{\textless{}{-}} \FunctionTok{aov}\NormalTok{(ROE }\SpecialCharTok{\textasciitilde{}}\NormalTok{ LForm, }\AttributeTok{data =}\NormalTok{ OB\_ES)}
\CommentTok{\# ANOVA por forma legal para Italia}
\NormalTok{anova\_roa\_it }\OtherTok{\textless{}{-}} \FunctionTok{aov}\NormalTok{(ROA }\SpecialCharTok{\textasciitilde{}}\NormalTok{ LForm, }\AttributeTok{data =}\NormalTok{ OB\_IT)}
\NormalTok{anova\_roe\_it }\OtherTok{\textless{}{-}} \FunctionTok{aov}\NormalTok{(ROE }\SpecialCharTok{\textasciitilde{}}\NormalTok{ LForm, }\AttributeTok{data =}\NormalTok{ OB\_IT)}
\end{Highlighting}
\end{Shaded}

Los resultados de los cálculos anteriores se combinan en una única
tabla. Se inicia con la tabla de promedios generales y, utilizando la
función left\_join(), se integran los resultados específicos de España e
Italia. Además, se renombraron las columnas para mejorar la claridad,
indicando a qué país o muestra total pertenecen los valores de ROE y
ROA. Finalmente, se redondean los valores a dos decimales mediante
mutate(across(where(is.numeric), \textasciitilde{} round(.x, 2))), lo
que facilita la presentación y comparación de los datos.

\begin{Shaded}
\begin{Highlighting}[]
\CommentTok{\# Combinar los datos de means }
\NormalTok{tabla\_means }\OtherTok{\textless{}{-}}\NormalTok{ legal\_form\_means\_GENERAL }\SpecialCharTok{\%\textgreater{}\%}
  \FunctionTok{rename}\NormalTok{(}\StringTok{"ROE\_Total\_sample"} \OtherTok{=}\NormalTok{ Mean\_ROE, }\StringTok{"ROA\_Total\_sample"} \OtherTok{=}\NormalTok{ Mean\_ROA) }\SpecialCharTok{\%\textgreater{}\%}
  \FunctionTok{left\_join}\NormalTok{(}
\NormalTok{    legal\_form\_means\_ESPAÑA }\SpecialCharTok{\%\textgreater{}\%}
      \FunctionTok{rename}\NormalTok{(}\StringTok{"ROE\_Spain"} \OtherTok{=}\NormalTok{ Mean\_ROE, }\StringTok{"ROA\_Spain"} \OtherTok{=}\NormalTok{ Mean\_ROA),}
    \AttributeTok{by =} \StringTok{"LForm"}
\NormalTok{  ) }\SpecialCharTok{\%\textgreater{}\%}
  \FunctionTok{left\_join}\NormalTok{(}
\NormalTok{    legal\_form\_means\_ITALIA }\SpecialCharTok{\%\textgreater{}\%}
      \FunctionTok{rename}\NormalTok{(}\StringTok{"ROE\_Italy"} \OtherTok{=}\NormalTok{ Mean\_ROE, }\StringTok{"ROA\_Italy"} \OtherTok{=}\NormalTok{ Mean\_ROA),}
    \AttributeTok{by =} \StringTok{"LForm"}
\NormalTok{  ) }\SpecialCharTok{\%\textgreater{}\%}
  \FunctionTok{mutate}\NormalTok{(}\FunctionTok{across}\NormalTok{(}\FunctionTok{where}\NormalTok{(is.numeric), }\SpecialCharTok{\textasciitilde{}} \FunctionTok{round}\NormalTok{(.x, }\DecValTok{2}\NormalTok{)))  }\CommentTok{\# Redondear a 2 decimales}
\end{Highlighting}
\end{Shaded}

Se genera un nuevo dataframe que almacena los resultados de los análisis
ANOVA. Se construye una tabla donde cada fila representa los valores de
F y p-value para ROE y ROA en la muestra total, España e Italia. Para
mejorar la interpretación de los resultados, los valores F se presentan
junto con una notación de significancia estadística (\emph{\textbf{, },
}), indicando niveles del 1\%, 5\% y 10\% respectivamente. Esta notación
se agrega utilizando estructuras condicionales ifelse(), que verifican
el valor p correspondiente a cada análisis de varianza.

\begin{Shaded}
\begin{Highlighting}[]
\CommentTok{\# Crear un dataframe separado para los valores ANOVA}
\NormalTok{anova\_results }\OtherTok{\textless{}{-}} \FunctionTok{data.frame}\NormalTok{(}
  \StringTok{\textasciigrave{}}\AttributeTok{Legal Form}\StringTok{\textasciigrave{}} \OtherTok{=} \FunctionTok{c}\NormalTok{(}\StringTok{"Total sample"}\NormalTok{, }\StringTok{"Spain"}\NormalTok{, }\StringTok{"Italy"}\NormalTok{),}
  
  \CommentTok{\# F{-}value con significancia para ROE}
  \StringTok{\textasciigrave{}}\AttributeTok{F{-}value (ROE)}\StringTok{\textasciigrave{}} \OtherTok{=} \FunctionTok{paste0}\NormalTok{(}
    \FunctionTok{round}\NormalTok{(}\FunctionTok{c}\NormalTok{(}
      \FunctionTok{summary}\NormalTok{(anova\_roe\_total)[[}\DecValTok{1}\NormalTok{]][[}\StringTok{"F value"}\NormalTok{]][}\DecValTok{1}\NormalTok{],}
      \FunctionTok{summary}\NormalTok{(anova\_roe\_es)[[}\DecValTok{1}\NormalTok{]][[}\StringTok{"F value"}\NormalTok{]][}\DecValTok{1}\NormalTok{],}
      \FunctionTok{summary}\NormalTok{(anova\_roe\_it)[[}\DecValTok{1}\NormalTok{]][[}\StringTok{"F value"}\NormalTok{]][}\DecValTok{1}\NormalTok{]}
\NormalTok{    ), }\DecValTok{2}\NormalTok{),}
    \FunctionTok{c}\NormalTok{(}
      \FunctionTok{ifelse}\NormalTok{(}\FunctionTok{summary}\NormalTok{(anova\_roe\_total)[[}\DecValTok{1}\NormalTok{]][[}\StringTok{"Pr(\textgreater{}F)"}\NormalTok{]][}\DecValTok{1}\NormalTok{] }\SpecialCharTok{\textless{}} \FloatTok{0.01}\NormalTok{, }\StringTok{"***"}\NormalTok{,}
             \FunctionTok{ifelse}\NormalTok{(}\FunctionTok{summary}\NormalTok{(anova\_roe\_total)[[}\DecValTok{1}\NormalTok{]][[}\StringTok{"Pr(\textgreater{}F)"}\NormalTok{]][}\DecValTok{1}\NormalTok{] }\SpecialCharTok{\textless{}} \FloatTok{0.05}\NormalTok{, }\StringTok{"**"}\NormalTok{,}
                    \FunctionTok{ifelse}\NormalTok{(}\FunctionTok{summary}\NormalTok{(anova\_roe\_total)[[}\DecValTok{1}\NormalTok{]][[}\StringTok{"Pr(\textgreater{}F)"}\NormalTok{]][}\DecValTok{1}\NormalTok{] }\SpecialCharTok{\textless{}} \FloatTok{0.10}\NormalTok{,}
                           \StringTok{"*"}\NormalTok{, }\StringTok{""}\NormalTok{)}
\NormalTok{             )}
\NormalTok{      ),}
      \FunctionTok{ifelse}\NormalTok{(}\FunctionTok{summary}\NormalTok{(anova\_roe\_es)[[}\DecValTok{1}\NormalTok{]][[}\StringTok{"Pr(\textgreater{}F)"}\NormalTok{]][}\DecValTok{1}\NormalTok{] }\SpecialCharTok{\textless{}} \FloatTok{0.01}\NormalTok{, }\StringTok{"***"}\NormalTok{,}
             \FunctionTok{ifelse}\NormalTok{(}\FunctionTok{summary}\NormalTok{(anova\_roe\_es)[[}\DecValTok{1}\NormalTok{]][[}\StringTok{"Pr(\textgreater{}F)"}\NormalTok{]][}\DecValTok{1}\NormalTok{] }\SpecialCharTok{\textless{}} \FloatTok{0.05}\NormalTok{, }\StringTok{"**"}\NormalTok{,}
                    \FunctionTok{ifelse}\NormalTok{(}\FunctionTok{summary}\NormalTok{(anova\_roe\_es)[[}\DecValTok{1}\NormalTok{]][[}\StringTok{"Pr(\textgreater{}F)"}\NormalTok{]][}\DecValTok{1}\NormalTok{] }\SpecialCharTok{\textless{}} \FloatTok{0.10}\NormalTok{,}
                           \StringTok{"*"}\NormalTok{, }\StringTok{""}\NormalTok{)}
\NormalTok{             )}
\NormalTok{      ),}
      \FunctionTok{ifelse}\NormalTok{(}\FunctionTok{summary}\NormalTok{(anova\_roe\_it)[[}\DecValTok{1}\NormalTok{]][[}\StringTok{"Pr(\textgreater{}F)"}\NormalTok{]][}\DecValTok{1}\NormalTok{] }\SpecialCharTok{\textless{}} \FloatTok{0.01}\NormalTok{, }\StringTok{"***"}\NormalTok{,}
             \FunctionTok{ifelse}\NormalTok{(}\FunctionTok{summary}\NormalTok{(anova\_roe\_it)[[}\DecValTok{1}\NormalTok{]][[}\StringTok{"Pr(\textgreater{}F)"}\NormalTok{]][}\DecValTok{1}\NormalTok{] }\SpecialCharTok{\textless{}} \FloatTok{0.05}\NormalTok{, }\StringTok{"**"}\NormalTok{,}
                    \FunctionTok{ifelse}\NormalTok{(}\FunctionTok{summary}\NormalTok{(anova\_roe\_it)[[}\DecValTok{1}\NormalTok{]][[}\StringTok{"Pr(\textgreater{}F)"}\NormalTok{]][}\DecValTok{1}\NormalTok{] }\SpecialCharTok{\textless{}} \FloatTok{0.10}\NormalTok{,}
                           \StringTok{"*"}\NormalTok{, }\StringTok{""}\NormalTok{)}
\NormalTok{             )}
\NormalTok{      )}
\NormalTok{    )}
\NormalTok{  ),}
  
  \CommentTok{\# p{-}value para ROE}
  \StringTok{\textasciigrave{}}\AttributeTok{p{-}value (ROE)}\StringTok{\textasciigrave{}} \OtherTok{=} \FunctionTok{c}\NormalTok{(}
    \FunctionTok{summary}\NormalTok{(anova\_roe\_total)[[}\DecValTok{1}\NormalTok{]][[}\StringTok{"Pr(\textgreater{}F)"}\NormalTok{]][}\DecValTok{1}\NormalTok{],}
    \FunctionTok{summary}\NormalTok{(anova\_roe\_es)[[}\DecValTok{1}\NormalTok{]][[}\StringTok{"Pr(\textgreater{}F)"}\NormalTok{]][}\DecValTok{1}\NormalTok{],}
    \FunctionTok{summary}\NormalTok{(anova\_roe\_it)[[}\DecValTok{1}\NormalTok{]][[}\StringTok{"Pr(\textgreater{}F)"}\NormalTok{]][}\DecValTok{1}\NormalTok{]}
\NormalTok{  ),}
  
  \CommentTok{\# F{-}value con significancia para ROA}
  \StringTok{\textasciigrave{}}\AttributeTok{F{-}value (ROA)}\StringTok{\textasciigrave{}} \OtherTok{=} \FunctionTok{paste0}\NormalTok{(}
    \FunctionTok{round}\NormalTok{(}\FunctionTok{c}\NormalTok{(}
      \FunctionTok{summary}\NormalTok{(anova\_roa\_total)[[}\DecValTok{1}\NormalTok{]][[}\StringTok{"F value"}\NormalTok{]][}\DecValTok{1}\NormalTok{],}
      \FunctionTok{summary}\NormalTok{(anova\_roa\_es)[[}\DecValTok{1}\NormalTok{]][[}\StringTok{"F value"}\NormalTok{]][}\DecValTok{1}\NormalTok{],}
      \FunctionTok{summary}\NormalTok{(anova\_roa\_it)[[}\DecValTok{1}\NormalTok{]][[}\StringTok{"F value"}\NormalTok{]][}\DecValTok{1}\NormalTok{]}
\NormalTok{    ), }\DecValTok{2}\NormalTok{),}
    \FunctionTok{c}\NormalTok{(}
      \FunctionTok{ifelse}\NormalTok{(}\FunctionTok{summary}\NormalTok{(anova\_roa\_total)[[}\DecValTok{1}\NormalTok{]][[}\StringTok{"Pr(\textgreater{}F)"}\NormalTok{]][}\DecValTok{1}\NormalTok{] }\SpecialCharTok{\textless{}} \FloatTok{0.01}\NormalTok{, }\StringTok{"***"}\NormalTok{,}
             \FunctionTok{ifelse}\NormalTok{(}\FunctionTok{summary}\NormalTok{(anova\_roa\_total)[[}\DecValTok{1}\NormalTok{]][[}\StringTok{"Pr(\textgreater{}F)"}\NormalTok{]][}\DecValTok{1}\NormalTok{] }\SpecialCharTok{\textless{}} \FloatTok{0.05}\NormalTok{, }\StringTok{"**"}\NormalTok{,}
                    \FunctionTok{ifelse}\NormalTok{(}\FunctionTok{summary}\NormalTok{(anova\_roa\_total)[[}\DecValTok{1}\NormalTok{]][[}\StringTok{"Pr(\textgreater{}F)"}\NormalTok{]][}\DecValTok{1}\NormalTok{] }\SpecialCharTok{\textless{}} \FloatTok{0.10}\NormalTok{, }
                           \StringTok{"*"}\NormalTok{, }\StringTok{""}\NormalTok{)}
\NormalTok{             )}
\NormalTok{      ),}
      \FunctionTok{ifelse}\NormalTok{(}\FunctionTok{summary}\NormalTok{(anova\_roa\_es)[[}\DecValTok{1}\NormalTok{]][[}\StringTok{"Pr(\textgreater{}F)"}\NormalTok{]][}\DecValTok{1}\NormalTok{] }\SpecialCharTok{\textless{}} \FloatTok{0.01}\NormalTok{, }\StringTok{"***"}\NormalTok{,}
             \FunctionTok{ifelse}\NormalTok{(}\FunctionTok{summary}\NormalTok{(anova\_roa\_es)[[}\DecValTok{1}\NormalTok{]][[}\StringTok{"Pr(\textgreater{}F)"}\NormalTok{]][}\DecValTok{1}\NormalTok{] }\SpecialCharTok{\textless{}} \FloatTok{0.05}\NormalTok{, }\StringTok{"**"}\NormalTok{,}
                    \FunctionTok{ifelse}\NormalTok{(}\FunctionTok{summary}\NormalTok{(anova\_roa\_es)[[}\DecValTok{1}\NormalTok{]][[}\StringTok{"Pr(\textgreater{}F)"}\NormalTok{]][}\DecValTok{1}\NormalTok{] }\SpecialCharTok{\textless{}} \FloatTok{0.10}\NormalTok{,}
                           \StringTok{"*"}\NormalTok{, }\StringTok{""}\NormalTok{)}
\NormalTok{             )}
\NormalTok{      ),}
      \FunctionTok{ifelse}\NormalTok{(}\FunctionTok{summary}\NormalTok{(anova\_roa\_it)[[}\DecValTok{1}\NormalTok{]][[}\StringTok{"Pr(\textgreater{}F)"}\NormalTok{]][}\DecValTok{1}\NormalTok{] }\SpecialCharTok{\textless{}} \FloatTok{0.01}\NormalTok{, }\StringTok{"***"}\NormalTok{,}
             \FunctionTok{ifelse}\NormalTok{(}\FunctionTok{summary}\NormalTok{(anova\_roa\_it)[[}\DecValTok{1}\NormalTok{]][[}\StringTok{"Pr(\textgreater{}F)"}\NormalTok{]][}\DecValTok{1}\NormalTok{] }\SpecialCharTok{\textless{}} \FloatTok{0.05}\NormalTok{, }\StringTok{"**"}\NormalTok{,}
                    \FunctionTok{ifelse}\NormalTok{(}\FunctionTok{summary}\NormalTok{(anova\_roa\_it)[[}\DecValTok{1}\NormalTok{]][[}\StringTok{"Pr(\textgreater{}F)"}\NormalTok{]][}\DecValTok{1}\NormalTok{] }\SpecialCharTok{\textless{}} \FloatTok{0.10}\NormalTok{,}
                           \StringTok{"*"}\NormalTok{, }\StringTok{""}\NormalTok{)}
\NormalTok{             )}
\NormalTok{      )}
\NormalTok{    )}
\NormalTok{  ),}
  
  \CommentTok{\# p{-}value para ROA}
  \StringTok{\textasciigrave{}}\AttributeTok{p{-}value (ROA)}\StringTok{\textasciigrave{}} \OtherTok{=} \FunctionTok{c}\NormalTok{(}
    \FunctionTok{summary}\NormalTok{(anova\_roa\_total)[[}\DecValTok{1}\NormalTok{]][[}\StringTok{"Pr(\textgreater{}F)"}\NormalTok{]][}\DecValTok{1}\NormalTok{],}
    \FunctionTok{summary}\NormalTok{(anova\_roa\_es)[[}\DecValTok{1}\NormalTok{]][[}\StringTok{"Pr(\textgreater{}F)"}\NormalTok{]][}\DecValTok{1}\NormalTok{],}
    \FunctionTok{summary}\NormalTok{(anova\_roa\_it)[[}\DecValTok{1}\NormalTok{]][[}\StringTok{"Pr(\textgreater{}F)"}\NormalTok{]][}\DecValTok{1}\NormalTok{]}
\NormalTok{  )}
\NormalTok{)}
\end{Highlighting}
\end{Shaded}

Se genera una fila adicional para la tabla de resultados con los valores
F y p-value, facilitando su integración con la tabla de promedios
previamente construida. Aquí, se formatean los valores para que cada
celda combine el estadístico F con su correspondiente p-value entre
paréntesis. Se hace uso de formatC() para mantener un formato de cuatro
decimales en los p-values, asegurando precisión en la presentación de
los resultados.

\begin{Shaded}
\begin{Highlighting}[]
\CommentTok{\# Crear una fila separada para los valores F y p{-}value con significancia}
\NormalTok{anova\_row }\OtherTok{\textless{}{-}} \FunctionTok{data.frame}\NormalTok{(}
  \AttributeTok{LForm =} \StringTok{"F"}\NormalTok{, }
  \StringTok{\textasciigrave{}}\AttributeTok{ROE\_Total\_sample}\StringTok{\textasciigrave{}} \OtherTok{=} \FunctionTok{paste0}\NormalTok{(}
\NormalTok{    anova\_results}\SpecialCharTok{$}\NormalTok{F.value..ROE.[}\DecValTok{1}\NormalTok{],}
    \StringTok{" ("}\NormalTok{, }\FunctionTok{formatC}\NormalTok{(anova\_results}\SpecialCharTok{$}\NormalTok{p.value..ROE.[}\DecValTok{1}\NormalTok{], }\AttributeTok{format =} \StringTok{"f"}\NormalTok{, }\AttributeTok{digits =} \DecValTok{4}\NormalTok{), }\StringTok{")"}
\NormalTok{  ),}
  \StringTok{\textasciigrave{}}\AttributeTok{ROA\_Total\_sample}\StringTok{\textasciigrave{}} \OtherTok{=} \FunctionTok{paste0}\NormalTok{(}
\NormalTok{    anova\_results}\SpecialCharTok{$}\NormalTok{F.value..ROA.[}\DecValTok{1}\NormalTok{],}
    \StringTok{" ("}\NormalTok{, }\FunctionTok{formatC}\NormalTok{(anova\_results}\SpecialCharTok{$}\NormalTok{p.value..ROA.[}\DecValTok{1}\NormalTok{], }\AttributeTok{format =} \StringTok{"f"}\NormalTok{, }\AttributeTok{digits =} \DecValTok{4}\NormalTok{), }\StringTok{")"}
\NormalTok{  ),}
  \StringTok{\textasciigrave{}}\AttributeTok{ROE\_Spain}\StringTok{\textasciigrave{}} \OtherTok{=} \FunctionTok{paste0}\NormalTok{(}
\NormalTok{    anova\_results}\SpecialCharTok{$}\NormalTok{F.value..ROE.[}\DecValTok{2}\NormalTok{],}
    \StringTok{" ("}\NormalTok{, }\FunctionTok{formatC}\NormalTok{(anova\_results}\SpecialCharTok{$}\NormalTok{p.value..ROE.[}\DecValTok{2}\NormalTok{], }\AttributeTok{format =} \StringTok{"f"}\NormalTok{, }\AttributeTok{digits =} \DecValTok{4}\NormalTok{), }\StringTok{")"}
\NormalTok{  ),}
  \StringTok{\textasciigrave{}}\AttributeTok{ROA\_Spain}\StringTok{\textasciigrave{}} \OtherTok{=} \FunctionTok{paste0}\NormalTok{(}
\NormalTok{    anova\_results}\SpecialCharTok{$}\NormalTok{F.value..ROA.[}\DecValTok{2}\NormalTok{],}
    \StringTok{" ("}\NormalTok{, }\FunctionTok{formatC}\NormalTok{(anova\_results}\SpecialCharTok{$}\NormalTok{p.value..ROA.[}\DecValTok{2}\NormalTok{], }\AttributeTok{format =} \StringTok{"f"}\NormalTok{, }\AttributeTok{digits =} \DecValTok{4}\NormalTok{), }\StringTok{")"}
\NormalTok{  ),}
  \StringTok{\textasciigrave{}}\AttributeTok{ROE\_Italy}\StringTok{\textasciigrave{}} \OtherTok{=} \FunctionTok{paste0}\NormalTok{(}
\NormalTok{    anova\_results}\SpecialCharTok{$}\NormalTok{F.value..ROE.[}\DecValTok{3}\NormalTok{],}
    \StringTok{" ("}\NormalTok{, }\FunctionTok{formatC}\NormalTok{(anova\_results}\SpecialCharTok{$}\NormalTok{p.value..ROE.[}\DecValTok{3}\NormalTok{], }\AttributeTok{format =} \StringTok{"f"}\NormalTok{, }\AttributeTok{digits =} \DecValTok{4}\NormalTok{), }\StringTok{")"}
\NormalTok{  ),}
  \StringTok{\textasciigrave{}}\AttributeTok{ROA\_Italy}\StringTok{\textasciigrave{}} \OtherTok{=} \FunctionTok{paste0}\NormalTok{(}
\NormalTok{    anova\_results}\SpecialCharTok{$}\NormalTok{F.value..ROA.[}\DecValTok{3}\NormalTok{],}
    \StringTok{" ("}\NormalTok{, }\FunctionTok{formatC}\NormalTok{(anova\_results}\SpecialCharTok{$}\NormalTok{p.value..ROA.[}\DecValTok{3}\NormalTok{], }\AttributeTok{format =} \StringTok{"f"}\NormalTok{, }\AttributeTok{digits =} \DecValTok{4}\NormalTok{), }\StringTok{")"}
\NormalTok{  )}
\NormalTok{)}
\end{Highlighting}
\end{Shaded}

Se realiza una transformación en las tablas previas para asegurar su
correcta visualización. Primero, las columnas numéricas se convierten en
caracteres y se reemplazan valores nulos con ``0'' mediante
replace\_na(). Luego, se combinan las tablas de promedios y valores
ANOVA en una única tabla consolidada (tabla\_combinada). Además, los
códigos numéricos de las formas legales de las empresas se reemplazan
por sus nombres correspondientes utilizando un diccionario de valores
(tabla5\_empresa), lo que mejora la interpretación de la información.

\begin{Shaded}
\begin{Highlighting}[]
\CommentTok{\# Convertir columnas en tabla\_means a character y reemplazar NA por "0"}
\NormalTok{tabla\_means }\OtherTok{\textless{}{-}}\NormalTok{ tabla\_means }\SpecialCharTok{\%\textgreater{}\%}
  \FunctionTok{mutate}\NormalTok{(}\FunctionTok{across}\NormalTok{(}\FunctionTok{everything}\NormalTok{(), as.character)) }\SpecialCharTok{\%\textgreater{}\%}
  \FunctionTok{mutate}\NormalTok{(}\FunctionTok{across}\NormalTok{(}\FunctionTok{everything}\NormalTok{(), }\SpecialCharTok{\textasciitilde{}} \FunctionTok{replace\_na}\NormalTok{(.x, }\StringTok{"0"}\NormalTok{)))}

\CommentTok{\# Convertir columnas en anova\_row a character y reemplazar NA por "0"}
\NormalTok{anova\_row }\OtherTok{\textless{}{-}}\NormalTok{ anova\_row }\SpecialCharTok{\%\textgreater{}\%}
  \FunctionTok{mutate}\NormalTok{(}\FunctionTok{across}\NormalTok{(}\FunctionTok{everything}\NormalTok{(), as.character)) }\SpecialCharTok{\%\textgreater{}\%}
  \FunctionTok{mutate}\NormalTok{(}\FunctionTok{across}\NormalTok{(}\FunctionTok{everything}\NormalTok{(), }\SpecialCharTok{\textasciitilde{}} \FunctionTok{replace\_na}\NormalTok{(.x, }\StringTok{"0"}\NormalTok{)))}

\CommentTok{\# Combinar ambas tablas}
\NormalTok{tabla\_combinada }\OtherTok{\textless{}{-}} \FunctionTok{bind\_rows}\NormalTok{(tabla\_means, anova\_row)}


\NormalTok{tabla5\_empresa }\OtherTok{\textless{}{-}} \FunctionTok{c}\NormalTok{(}\StringTok{"0"} \OtherTok{=} \StringTok{"Public limited company"}\NormalTok{, }
                    \StringTok{"1"} \OtherTok{=} \StringTok{"Private limited company"}\NormalTok{, }
                    \StringTok{"3"} \OtherTok{=} \StringTok{"Cooperative"}\NormalTok{, }
                    \StringTok{"4"} \OtherTok{=} \StringTok{"Other legal forms"}\NormalTok{,}
                    \StringTok{"F"} \OtherTok{=} \StringTok{"F"}\NormalTok{)}
\CommentTok{\# Reemplazar los códigos numéricos con los nombres de tipos de empresas}
\NormalTok{tabla\_combinada }\OtherTok{\textless{}{-}}\NormalTok{tabla\_combinada }\SpecialCharTok{\%\textgreater{}\%}
  \FunctionTok{mutate}\NormalTok{(}\AttributeTok{LForm =}\NormalTok{ tabla5\_empresa[}\FunctionTok{as.character}\NormalTok{(LForm)])}\SpecialCharTok{\%\textgreater{}\%} 
  \FunctionTok{rename}\NormalTok{(}\StringTok{"Legal\_Form"} \OtherTok{=}\NormalTok{ LForm)}
\end{Highlighting}
\end{Shaded}

Finalmente, se genera una tabla en formato visual utilizando la librería
gt(). Se establecen títulos y subtítulos con formato Markdown para
mejorar la presentación del informe. Se organizan las columnas bajo tres
categorías: muestra total, España e Italia, utilizando tab\_spanner().Se
configuran estilos de negrita para los grupos de filas y se alinean los
encabezados y datos en el centro. Además, se añaden notas al pie para
clarificar que los valores p aparecen entre paréntesis y que los niveles
de significancia se indican con asteriscos. Finalmente, se ajustan los
tamaños de fuente mediante tab\_options(), asegurando que la tabla sea
clara y legible.

\begin{Shaded}
\begin{Highlighting}[]
\CommentTok{\# Crear la tabla con \textasciigrave{}gt()\textasciigrave{}}
\NormalTok{tabla\_combinada }\SpecialCharTok{\%\textgreater{}\%}
  \FunctionTok{gt}\NormalTok{(}\AttributeTok{rowname\_col =} \StringTok{"Forma Legal"}\NormalTok{) }\SpecialCharTok{\%\textgreater{}\%}
  \FunctionTok{tab\_header}\NormalTok{(}
    \AttributeTok{title =} \FunctionTok{md}\NormalTok{(}\StringTok{"**Tabla 5. Promedio de ROA y ROE por forma legal**"}\NormalTok{),}
    \AttributeTok{subtitle =} \FunctionTok{md}\NormalTok{(}\StringTok{"Análisis de varianza."}\NormalTok{)}
\NormalTok{  ) }\SpecialCharTok{\%\textgreater{}\%}
  \FunctionTok{cols\_label}\NormalTok{(}
    \AttributeTok{Legal\_Form =} \StringTok{"FORMA LEGAL"}\NormalTok{,}
    \AttributeTok{ROE\_Total\_sample =} \StringTok{"ROE"}\NormalTok{,}
    \AttributeTok{ROA\_Total\_sample =} \StringTok{"ROA"}\NormalTok{,}
    \AttributeTok{ROE\_Spain =} \StringTok{"ROE"}\NormalTok{,}
    \AttributeTok{ROA\_Spain =} \StringTok{"ROA"}\NormalTok{,}
    \AttributeTok{ROE\_Italy =} \StringTok{"ROE"}\NormalTok{,}
    \AttributeTok{ROA\_Italy =} \StringTok{"ROA"}
\NormalTok{  ) }\SpecialCharTok{\%\textgreater{}\%}
  \FunctionTok{tab\_spanner}\NormalTok{(}
    \AttributeTok{label =} \StringTok{"Muestra total"}\NormalTok{,}
    \AttributeTok{columns =} \FunctionTok{c}\NormalTok{(ROE\_Total\_sample, ROA\_Total\_sample)}
\NormalTok{  ) }\SpecialCharTok{\%\textgreater{}\%}
  \FunctionTok{tab\_spanner}\NormalTok{(}
    \AttributeTok{label =} \StringTok{"España"}\NormalTok{,}
    \AttributeTok{columns =} \FunctionTok{c}\NormalTok{(ROE\_Spain, ROA\_Spain)}
\NormalTok{  ) }\SpecialCharTok{\%\textgreater{}\%}
  \FunctionTok{tab\_spanner}\NormalTok{(}
    \AttributeTok{label =} \StringTok{"Italia"}\NormalTok{,}
    \AttributeTok{columns =} \FunctionTok{c}\NormalTok{(ROE\_Italy, ROA\_Italy)}
\NormalTok{  ) }\SpecialCharTok{\%\textgreater{}\%}
  \FunctionTok{tab\_style}\NormalTok{(}
    \AttributeTok{style =} \FunctionTok{list}\NormalTok{(}
      \FunctionTok{cell\_text}\NormalTok{(}\AttributeTok{weight =} \StringTok{"bold"}\NormalTok{)}
\NormalTok{    ),}
    \AttributeTok{locations =} \FunctionTok{cells\_row\_groups}\NormalTok{()}
\NormalTok{  ) }\SpecialCharTok{\%\textgreater{}\%}
  \FunctionTok{opt\_table\_lines}\NormalTok{(}\AttributeTok{extent =} \StringTok{"all"}\NormalTok{) }\SpecialCharTok{\%\textgreater{}\%}
  \FunctionTok{opt\_align\_table\_header}\NormalTok{(}\AttributeTok{align =} \StringTok{"center"}\NormalTok{) }\SpecialCharTok{\%\textgreater{}\%}
  \FunctionTok{cols\_align}\NormalTok{(}
    \AttributeTok{align =} \StringTok{"center"}\NormalTok{,}
    \AttributeTok{columns =} \FunctionTok{everything}\NormalTok{()}
\NormalTok{  ) }\SpecialCharTok{\%\textgreater{}\%}
  \FunctionTok{tab\_footnote}\NormalTok{(}
    \AttributeTok{footnote =} \FunctionTok{md}\NormalTok{(}\StringTok{"*p{-}valor entre paréntesis.*"}\NormalTok{),}
    \AttributeTok{locations =} \FunctionTok{cells\_title}\NormalTok{(}\AttributeTok{groups =} \StringTok{"subtitle"}\NormalTok{)}
\NormalTok{  ) }\SpecialCharTok{\%\textgreater{}\%}
  \FunctionTok{tab\_footnote}\NormalTok{(}
    \AttributeTok{footnote =} \FunctionTok{md}\NormalTok{(}\StringTok{"***, ** y * denotan un nivel de significancia}
\StringTok{                  por debajo del 1\%, 5\% y 10\%, respectivamente.*"}\NormalTok{),}
    \AttributeTok{locations =} \FunctionTok{cells\_title}\NormalTok{(}\AttributeTok{groups =} \StringTok{"subtitle"}\NormalTok{)}
\NormalTok{  ) }\SpecialCharTok{\%\textgreater{}\%}
  \FunctionTok{tab\_source\_note}\NormalTok{(}
    \AttributeTok{source\_note =} \StringTok{"Fuente: Elaboración propia."}
\NormalTok{  ) }\SpecialCharTok{\%\textgreater{}\%}
  \FunctionTok{tab\_options}\NormalTok{(}
    \AttributeTok{column\_labels.font.size =} \FunctionTok{px}\NormalTok{(}\DecValTok{10}\NormalTok{), }\CommentTok{\# Tamaño del texto del encabezado}
    \AttributeTok{table.font.size =} \FunctionTok{px}\NormalTok{(}\DecValTok{10}\NormalTok{) }\CommentTok{\# Tamaño del texto en la tabla}
\NormalTok{  )}\SpecialCharTok{\%\textgreater{}\%}
  \FunctionTok{tab\_style}\NormalTok{(}
    \AttributeTok{style =} \FunctionTok{list}\NormalTok{(}
      \FunctionTok{cell\_text}\NormalTok{(}\AttributeTok{align =} \StringTok{"center"}\NormalTok{)  }
\NormalTok{    ),}
    \AttributeTok{locations =} \FunctionTok{cells\_title}\NormalTok{(}\AttributeTok{groups =} \StringTok{"title"}\NormalTok{) }
\NormalTok{  )}
\end{Highlighting}
\end{Shaded}

\begin{table}[!t]
\caption*{
{\large \textbf{Tabla 5. Promedio de ROA y ROE por forma legal}} \\ 
{\small Análisis de varianza.\textsuperscript{\textit{1,2}}}
} 
\fontsize{7.5pt}{9.0pt}\selectfont
\begin{tabular*}{\linewidth}{@{\extracolsep{\fill}}ccccccc}
\toprule
 & \multicolumn{2}{c}{Muestra total} & \multicolumn{2}{c}{España} & \multicolumn{2}{c}{Italia} \\ 
\cmidrule(lr){2-3} \cmidrule(lr){4-5} \cmidrule(lr){6-7}
FORMA LEGAL & ROE & ROA & ROE & ROA & ROE & ROA \\ 
\midrule\addlinespace[2.5pt]
Public limited company & 2.67 & 1.62 & 2.17 & 1.39 & 4.69 & 2.5 \\ 
Private limited company & 5.5 & 3.4 & 3.12 & 3.75 & 6.9 & 3.18 \\ 
Cooperative & 4.46 & 1.64 & 2.91 & 1.39 & 5.05 & 1.74 \\ 
Other legal forms & -0.99 & 0.73 & 0 & 0 & -0.99 & 0.73 \\ 
F & 0.48 (0.6989) & 6.2*** (0.0003) & 0.03 (0.9709) & 7.75*** (0.0005) & 0.38 (0.7666) & 1.6 (0.1870) \\ 
\bottomrule
\end{tabular*}
\begin{minipage}{\linewidth}
\textsuperscript{\textit{1}}\emph{p-valor entre paréntesis.}\\
\textsuperscript{\textit{2}}**\emph{, ** y * denotan un nivel de significancia
por debajo del 1\%, 5\% y 10\%, respectivamente.}\\
Fuente: Elaboración propia.\\
\end{minipage}
\end{table}

\end{document}
